\documentclass[a4paper,12pt]{article}

%% Начало шапки

%% Настройка поддержки русского языка
\usepackage{cmap}                   % Поиск по кириллице
\usepackage{mathtext}               % Кириллица в формулах
\usepackage[T1,T2A]{fontenc}        % Кодировки шрифтов
\usepackage[utf8]{inputenc}         % Кодировка текста
\usepackage[english,russian]{babel} % Подключение поддержки языков

%% Настройка размеров полей
\usepackage[top=0.7in, bottom=0.75in, left=0.625in, right=0.625in]{geometry}

%% Математические пакеты
\usepackage{mathtools}              % Тот же amsmath, только с некоторыми поправками
\usepackage{amssymb}                % Математические символы
\usepackage{amsthm}                 % Оформление теорем
\usepackage{amstext}                % Текстовые вставки в формулы
\usepackage{amsfonts}               % Математические шрифты
\usepackage{icomma}                 % "Умная" запятая: $0,2$ --- число, $0, 2$ --- перечисление
\usepackage{enumitem}               % Для выравнивания itemize (\begin{itemize}[align=left])
\usepackage{array}                  % Таблицы и матрицы
\usepackage{multirow}

%% Алгоритмические пакеты и их настройки
\usepackage{algorithm}              % Шапка алгоритма
\usepackage{algorithmicx}           % Написание алгоритмов
\usepackage[noend]{algpseudocode}   % Написание псевдокода; убраны end
\usepackage{listings}               % Для кода на каком-либо языке программиования
\renewcommand{\algorithmicrequire}{\textbf{Ввод:}}              % Ввод
\renewcommand{\algorithmicensure}{\textbf{Вывод:}}              % Вывод
\floatname{algorithm}{Алгоритм}                                 % Название алгоритма
\renewcommand{\algorithmiccomment}[1]{\hspace*{\fill}\{// #1\}} % Комментарии
\newcommand{\algname}[1]{\textsc{#1}}                           % Вызов функции в алгоритме

\newcommand*{\hm}[1]{#1\nobreak\discretionary{}
	{\hbox{$\mathsurround=0pt #1$}}{}}

%% Шрифты
\usepackage{euscript}               % Шрифт Евклид
\usepackage{mathrsfs}               % \mathscr{}

%% Графика
\usepackage[pdftex]{graphicx}       % Вставка картинок
\graphicspath{{images/}}            % Стандартный путь к картинкам
\usepackage{tikz}                   % Рисование всего
\usepackage{pgfplots}               % Графики
\usetikzlibrary{calc,matrix}

%% Прочие пакеты
\usepackage{indentfirst}                    % Красная строка в начале текста
\usepackage{epigraph}                       % Эпиграфы
\usepackage{fancybox,fancyhdr}              % Колонтитулы
\usepackage[colorlinks=true, urlcolor=blue]{hyperref}   % Ссылки
\usepackage{titlesec}                       % Изменение формата заголовков
\usepackage[normalem]{ulem}                 % Для зачёркиваний
\usepackage[makeroom]{cancel}               % И снова зачёркивание (на этот раз косое)

%% Прочее
\mathtoolsset{showonlyrefs=true}        % Показывать номера только у тех формул,
% на которые есть \eqref{} в тексте.
\renewcommand{\headrulewidth}{1.8pt}    % Изменяем размер верхнего отступа колонтитула
\renewcommand{\footrulewidth}{0.0pt}    % Изменяем размер нижнего отступа колонтитула

%Прочее
\usepackage{forest} % Деревья

\renewcommand{\Re}{\mathrm{Re\:}}
\renewcommand{\Im}{\mathrm{Im\:}}
\newcommand{\Arg}{\mathrm{Arg\:}}
\renewcommand{\arg}{\mathrm{arg\:}}
\newcommand{\Mat}{\mathrm{Mat}}
\newcommand{\M}{\mathrm{M}}
\newcommand{\id}{\mathrm{id}}
\newcommand{\isom}{\xrightarrow{\sim}} 
\newcommand{\leftisom}{\xleftarrow{\sim}}
\newcommand{\Hom}{\mathrm{Hom}}
\newcommand{\Ker}{\mathrm{Ker}\:}
\newcommand{\rk}{\mathrm{rk}\:}
\newcommand{\diag}{\mathrm{diag}}
\newcommand{\ort}{\mathrm{ort}}
\newcommand{\pr}{\mathrm{pr}}
\newcommand{\vol}{\mathrm{vol\:}}
\newcommand{\Tr}{\mathrm{tr\:}}
\newcommand{\sgn}{\mathrm{sgn\:}}
\newcommand{\al}{\alpha}

%% Определения
\newtheorem{definition}{Определение}
\newtheorem*{defin}{Определение}
\newtheorem{Def}{Определение}
\newtheorem*{Lemma}{Лемма}
\newtheorem{Suggestion}{Предложение}
\newtheorem*{Examples}{Пример}
\newtheorem*{Consequence}{Следствие}
\newtheorem{Theorem}{Теорема}
\newtheorem*{ther}{Теорема}
\newtheorem{Statement}{Утверждение}
\newtheorem*{Task}{Упражнение}
\newtheorem*{Designation}{Обозначение}
\newtheorem*{Generalization}{Обобщение}
\newtheorem*{Thedream}{Предел мечтаний}
\newtheorem*{Properties}{Свойства}
\newtheorem*{Note}{Замечание}

\newcommand{\Z}{\mathbb{Z}}
\newcommand{\N}{\mathbb{N}}
\newcommand{\Q}{\mathbb{Q}}
\newcommand{\R}{\mathbb{R}}
\renewcommand{\C}{\mathbb{C}}
\renewcommand{\L}{\mathscr{L}}
\renewcommand{\epsilon}{\varepsilon}
\renewcommand{\phi}{\varphi}
\newcommand{\e}{\mathbb{e}}
\renewcommand{\l}{\lambda}
\newcommand{\E}{\mathbb{E}}
\def\eps{\varepsilon}
\def\limref#1#2{{#1}\negmedspace\mid_{#2}}
\newcommand{\vvector}[1]{\begin{pmatrix}{#1}_1 \\\vdots\\{#1}_n\end{pmatrix}}
\renewcommand{\vector}[1]{({#1}_1, \ldots, {#1}_n)}

\definecolor{Gray}{gray}{0.9}
\newcolumntype{g}{>{\columncolor{Gray}}c}

\newtheorem*{canonther}{Теорема о приведении матрицы к каноническому виду}
\newtheorem*{lem}{Лемма}
\newtheorem*{lem1}{Лемма 1}
\newtheorem*{lem2}{Лемма 2}
\newtheorem*{lem3}{Лемма 3}
\newcommand{\divby}{\ \tikz \foreach \y in {0ex, 0.65ex, 1.3ex} \fill (0,\y) circle (0.5pt);\ }



\begin{document}
	\title{Материалы для подготовки к коллоквиуму\\ по дискретной математике \\ Определения}
	\author{ПМИ 2016 \\ Орлов Никита, Рубачев Иван, Ткачев Андрей, Евсеев Борис}
	\maketitle

	\section*{Принцип математической индукции}
	Принципом математической индукции называют метод доказательства бесконечной цепочки утверждений, пронумерованных натуральными числами. Тогда для их доказательства достаточно справедливости следующих фактов:
	\begin{enumerate}
		\item Верно утверждение $A(0)$, называемое \textit{базой индукции}.
		\item Для любого натурального $n$ верно, что из $A(n)$ следует $A(n + 1)$. Этот переход называется \textit{шагом индукции}.
	\end{enumerate}
	
	В качетсве примера может служить доказательство формул арифметической, геометрической прогрессий, коды Грея.
	
	\begin{center}
		\line(1,0){450}
	\end{center}

	\section*{Правила суммы, произведения, дополнения}
	Пусть есть непересекающиеся множества $A, B$. Тогда
	\[
	|A \cup B| = |A| + |B|,
	\]
	\[
	|A \times B| = |A| \times |B|
	\]
	называются правилами суммы и произведения множеств соответственно.
	
	\textit{Дополнением} $\overline{A}$ множества $A$ называется множество, состоящее из не удовлетворяющих произвольному условию элементов. Тогда
	\[
	|\overline{A}| = U - A,
	\]
	где $U$ - пространство, в котором решается задача.
	
	\begin{center}
		\line(1,0){450}
	\end{center}

	\section*{Алфавит, конечные слова, формулы комбинаторики}
	\textit{Алфавитом} называется произвольное конечное множество, элементы которого называются символами или буквами.
	
	\textit{Словом} называется произвольная упорядоченная последовательность букв.
	
	\textit{Числом перестановок} $n!$ слова называется количество слов длины $n$, отличающихся друг от друга порядком следования букв.
	\[
	n! = 1 \cdot 2 \cdot \ldots \cdot (n - 1) \cdot n
	\]
	
	\textit{Упорядоченным выбором с возвращением} из $n$ по $k$ называется слово длины $k$, состоящее из букв слова длины $n$, с повторяющимися буквами. Число таких слов будет равняться 
	\[
	n^k
	\]
	
	\textit{Упорядоченным выбором без возвращения} из $n$ по $k$ называется слово длины $k$, состоящее из букв слова длины $n$, без повторяющихся букв. Число таких слов будет равняться
	\[
	A_n^k = \frac{n!}{(n - k)!}
	\]
	
	\textit{Неупорядоченным выбором без возвращения} из $n$ по $k$ называется слово длины $k$, состоящее из букв слова длины $n$, без повторяющихся букв, причем слова, отличающиеся только порядком следования букв будем считать одинаковыми. Тогда число таких слов будет равняться 
	\[
	C_n^k = \frac{n!}{k! \ (n - k)!} = {n \choose k}
	\]
	
	\textit{Неупорядоченным выбором с возвращением} ..........
	\[
	{k - 1 \choose n + k - 1} = {n - 1 \choose n + k - 1}
	\]
	
	\textit{Двоичными словами} называются слова, составленные из двух букв, называемых \textit{нулем} и \textit{единицей}:
	\[
	a \in \{0; 1\}
	\]
	

	\textit{Число подмножеств} множества считается по формуле 
	\[
	2 ^ {|A|},
	\]
	где $A$ - множество.
	\begin{center}
		\line(1,0){450}
	\end{center}

	\section*{Формула включений-исключений}
	\textit{Формулой включений-исключений} называется формула, по которой можно посчитать мощность объединения счетного количества множеств:
	\[
	|A_1 \cup \ldots \cup A_n| = \sum_{k = 1}^{n} -1^{(k + 1)} \cdot (\sum_{1 < m_1 < \ldots <m_k \leq n} |A_{m_1} \cap \ldots \cap A_{m_k}|)
	\]
	
	\begin{center}
		\line(1,0){450}
	\end{center}
	
	\section*{Биноминальные коэффициенты, основные свойства. Бином Ньютона}
	\textit{Биноминальными коэффициентами} называются коэффициенты в разложении бинома Ньютона $(1 + x)^n$ по степеням $x$. При $k$ степени $x$ -- ${n \choose k}$.
	
	\textit{Бином Ньютона} - формула разложения степени двучлена в сумму:
	\[
	(a + b)^n = \sum_{k = 0}^{n} {n \choose k} a^{n - k}b^k
	\]
	
	Свойства биноминальных коэффициентов:
	\begin{enumerate}
		\item ${n \choose k} = {n - k \choose k}$.
		\item ${n \choose k} = {n - 1 \choose k} + {n -1 \choose k - 1}.$
		\item $\sum_{k = 0}^n {n \choose k} = 2 ^ n$
	\end{enumerate}
	
	
	\begin{center}
		\line(1,0){450}
	\end{center}
	
	\section*{Треугольник Паскаля. Реккурентное соотношение}
	\textit{Реккурентным соотношением} называется формула, где каждый следующий член определен через предыдущие числа. Пример: числа Фибонначи. 
	
	\textit{Треугольник Паскаля} - треугольник биномиальных коэффициентов, где каждый следующий элемент определяется суммой двух элементов над ним:
	\[
	\begin{matrix}
			&	&	&	&1	&	&	&	\\
			&	&	&1	&	& 1 & 	&	\\
			&	&1	&	&2	&	&1 	& 	\\	
			&1	&	&3	&	&3	& 	&1 	\\	
			
	\end{matrix}	
	\]
	
	\section*{Графы}
	Пусть у нас есть множество элементов $V$ - множество \textit{вершин}. 
	\textit{Граф} - математический объект, являющийся совокупностью \textit{вершин} и \textit{ребер}, то есть:
	\[
	G := (V, E), \ E \subseteq V \times V
	\]
	
	\textit{Ребром} графа называется пара вершин.
	
	\textit{Неориентированный} гарф - такой граф, где ребрам не задано направление.
	
	\textit{Ориентированный} гарф - такой граф, где у каждого ребра есть направление, иными словами, у каждого ребра есть начальная и конечная вершины.
	
	\textit{Матрица смежности} - квадратная матрица размера $V \times V$, строки и столбцы одинаково пронумерованы, элемент $a_{ij}$ показывает наличие ребра или его вес:
	\begin{center}
		\begin{tabular}{c|c|c|c|c}
			&1	&2	&3	&4 	\\ \hline
			1	&$a_{11}$	&$a_{12}$	&$a_{13}$	&$a_{14}$ \\ \hline
			2	&$a_{21}$	&$a_{22}$	&$a_{23}$	&$a_{24}$	\\ \hline
			3	&$a_{31}$	&$a_{32}$	&$a_{33}$	&$a_{34}$	\\ \hline
			4	&$a_{41}$	&$a_{42}$	&$a_{43}$	&$a_{44}$ 
		\end{tabular}
	\end{center}
	
	
	
	\textit{Изоморфные графы} - такие графы $G$ и $G'$, что можно построить биекцию между их вершинами и соответствующими ребрами, или их вершины можно перенумеровать так, что их матрицы смежности совпадут
	
	\textit{Cтепень вершины} - число ребер, исходящих из нее, причем сумма степеней вершин равна удвоенному числу ребер.
	
	\begin{center}
		\line(1,0){450}
	\end{center}
	
	\section*{Пути и циклы в графах}
	\textit{Маршрут} - последовательность ребер, т.ч. соседние ребра имеют общий конец.
	
	\textit{Путь} - маршрут без повторений ребер.
	
	\textit{Простой путь} - путь без повторения вершин.
	
	\textit{Цикл} - путь, в котором первая и последняя вершины совпадают 
	
	\textit{Простой цикл} - цикл, в котором все вершины, кроме начальной и конечной, различны.
	
	\textit{Длина пути/цикла} - число ребер, в них входящих.
	
	\section*{Отношение связанности и компоненты связности графа}
	\textit{Связность} - граф связен тогда и только тогда, когда две любые вершины соединены путем.
	
	\textit{Компонента связности} - максимальный по включению связный подграф графа $G$ -- $G(U)$, порожденный подмножеством вершин исходного графа, в котором для любой пары $v_1, v_2 \in U$ существует путь, а для всех других пар пути нет.
	
	\begin{center}
		\line(1,0){450}
	\end{center}
	
	\section*{Дерево. Примеры. Полные бинарные деревья}
	\textit{Дерево} - минимальный связный ациклический граф. В нем любые две вершины соединены единственным путем. 
	
	Во всяком связном графе существует \textit{остовное дерево} --- подграф--дерево, содержащий все вершины.
	
	Пример дерева:
	\begin{center}
		\begin{tikzpicture}  
		\node [circle, black, draw=black, fill=black!20, minimum size=0.5cm] (v2) at (0,2) {};
		\node [circle, black, draw=black, fill=black!20, minimum size=0.5cm] (v1) at (-1.5,0.5) {};
		\node [circle, black, draw=black, fill=black!20, minimum size=0.5cm] (v3) at (1.5,0.5) {};
		\node [circle, black, draw=black, fill=black!20, minimum size=0.5cm] (v7) at (-2.5,-1) {};
		\node [circle, black, draw=black, fill=black!20, minimum size=0.5cm] (v6) at (-0.5,-1) {};
		\node [circle, black, draw=black, fill=black!20, minimum size=0.5cm] (v4) at (2.5,-1) {};
		\node [circle, black, draw=black, fill=black!20, minimum size=0.5cm] (v5) at (0.5,-1) {};
		\draw  (v1) edge (v2);
		\draw  (v2) edge (v3);
		\draw  (v3) edge (v4);
		\draw  (v3) edge (v5);
		\draw  (v1) edge (v6);
		\draw  (v1) edge (v7);
		\end{tikzpicture}
	\end{center}
	
	\textit{Полное бинарное дерево} - дерево, вершинами которого являются бинарные слова, а ребра получаются приписыванием 0 или 1 в конец предыдущего слова (слова - родителя):
	
	\begin{center}
		\begin{tikzpicture}
			\node  (v2) at (0,2) {1};
			\node  (v1) at (-1.5,0.5) {10};
			\node (v3) at (1.5,0.5) {11};
			\node  (v7) at (-2.5,-1) {100};
			\node (v6) at (-0.5,-1) {101};
			\node (v4) at (2.5,-1) {111};
			\node  (v5) at (0.5,-1) {110};
			\draw  (v1) edge (v2);
			\draw  (v2) edge (v3);
			\draw  (v3) edge (v4);
			\draw  (v3) edge (v5);
			\draw  (v1) edge (v6);
			\draw  (v1) edge (v7);
		\end{tikzpicture}
	\end{center}
	
	\begin{center}
		\line(1,0){450}
	\end{center}

	\section*{Правильные раскраски графов}
	\textit{Правильной раскраской} графа называется такой способ пронумеровать(покрасить) вершины, что никакие две вершины одинакого номера(цвета) не соединены ребром.
	
	\textit{Двудольный}, или \textit{двураскрашиваемый} граф - граф, вершины которого можно разбить на два непересекающихся подмножества, таких что ни одно ребро не соединяет вершины, лежащие в одном подмножестве.
	
	Граф двураскрашиваем тогда и только тогда, когда в нем нет цикла нечетной длины.
	
	\begin{center}
		\line(1,0){450}
	\end{center}

	\section*{Ориентированный граф}
	\textit{Ориентированный граф} - такой граф, на ребрах которого установлено направление обхода.
	
	\textit{Маршрут} в орграфе - чередующаяся последовательность вершин и дуг, где вершины могут повторяться.
	\textit{Путь} в орграфе - маршрут без повторяющихся дуг, \textit{простой путь} - без повторяющихся вершин. Если существует путь из одной вершины в другую, тогда говорят, что вторая достижима из первой.
	
	\begin{center}
		\line(1,0){450}
	\end{center}

	\section*{Компоненты сильной связности}
	\textit{Компонентой сильной связности} называют подграф, в котором любая вершина достижима из любой другой.
	\textit{Ациклический орграф} - такой граф, в котором нет циклов, и все компоненты сильной связности состоят из одной вершины.
	
	\begin{center}
		\line(1,0){450}
	\end{center}

	\section*{Эйлеровы и гамильтоновы циклы}
	\textit{Эйлеров цикл} - цикл, проходящий через каждое ребро графа ровно по одному разу. Он существует тогда, когда степени всех вершин четны.
	\textit{Гамильтонов цикл} - цикл, проходящий через каждую вершину графа по одному разу.
	
	\section*{Делимость целых чисел}
	Говорят, что $a$ делится на $b$, или $b$ делит $a$, если существует такое $k$, что
	\[
	a \divby b = b | a \rightarrow a = kb
	\]
	
	Свойства: ..............
	
	\begin{center}
		\line(1,0){450}
	\end{center}
	
	\section*{Деление целых чисел с остатком}
	\textit{Остатком деления числа a на b} называется такое число $r$, что 
	\[
	a = kb + r, \ 0 \leq r < b
	\]
	Частное и остаток определены однозначно для всех пар чисел.
	
	\begin{center}
		\line(1,0){450}
	\end{center}

	\section*{Сравнения по модулю}
	Два числа \textit{сравнимы по модулю}, если их остатки при делении на число, называемое \textit{модулем} совпадают.
	\[
	a \equiv b \ mod \ c
	\]
	Эта запись означает, что остатки от деления $a$ и $b$ на $c$ равны.
	
	Основные свойства:
	\begin{enumerate}
		\item $a \equiv b \ mod \ c$ и $d \equiv e \ mod \ c \Rightarrow (a + d) \equiv (b + e) \ mod \ c$
		\item $a \equiv b \ mod \ c$ и $d \equiv e \ mod \ c \Rightarrow (ad) \equiv (be) \ mod \ c$
	
	\end{enumerate}
	
	\begin{center}
		\line(1,0){450}
	\end{center}
	
	\section*{17. Арифметика остатков}
	
	\begin{center}
		\line(1,0){450}
	\end{center}

	\section*{18. Малая теорема Ферма. Лемма Вильсона}
	\textit{Малая теорема Ферма} гласит о том, что если $p > 2$ - простое число, $a$ - целое число, не делящееся на $p$, то
	\[
	a^{(p-1)} \equiv 1 \ mod \ p
	\]
	
	\textit{Лемма Вильсона} гласит о том, что число $p$ простое тогда и только тогда, когда 
	\[
	(p - 1)! \equiv -1 \ mod \ p
	\]
	
	\begin{center}
		\line(1,0){450}
	\end{center}

	\section*{19. Функция Эйлера. Теорема Эйлера}
	\textit{Функция Эйлера} $\phi(n)$ возвращает количество чисел, меньших $n$ и взаимно простых с ним.
	
	\textit{Теорема Эйлера} утверждает, что если $a$ и $m$ взаимно просты, то 
	\[
	a^{\phi(n)} \equiv 1 \ (mod \ m)
	\]
	
	\begin{center}
		\line(1,0){450}
	\end{center}

	\section*{20. Наибольший общий делитель. Алгоритм Евклида}
	\textit{Наибольшим общим делителем} двух чисел $a$ и $b$ называют такое наибольшее число $c$, что $c|a$ и $c|b$.
	
	\textit{Алгоритм Евклида} - итеративный алгоритм, который ищет НОД двух чисел. Он состоит в следующем:
	\begin{enumerate}
		\item Вычитаем из большего числа меньшее
		\item Заменяем большее на полученную разность
		\item Повторяем 1 -- 2 до тех пор, пока не получим равные числа. Если числа равны, то говорим, что последнее полученное таким образом число и есть наибольший общий делитель.
	\end{enumerate}

	\begin{center}
		\line(1,0){450}
	\end{center}

	\section*{21. Расширенный алгоритм Евклида}
	\textit{Линейное диофантово уравнение} - уравнение вида $ax + by = c$, где $a, b, c$ - коэффициенты, $x, y$ - неизвестные.
	
	\textit{Расширенный алгоритм Евклида} ищет решения линейного диофантова уравнения.
	...............
	
	\begin{center}
		\line(1,0){450}
	\end{center}

	\section*{22. Простые числа. Формулировка основной теоремы арифметики.}
	\textit{Простое число} - большее единицы число, такое что оно делится только на 1 и на себя.
	
	\textit{Основная теорема арифметики} гласит о том, что всякое число представимо в виде произведения простых, причем такое представление единственно с точностью до порядка следования сомножителей.

	\begin{center}
		\line(1,0){450}
	\end{center}

	\section*{23. Бинарные отношения и операции над ними}
	\textit{Бинарным отношением над множествами $A$ и $B$} называется множество $P \subseteq A \times B$. Элементы этого множества суть пары, которые определяют, состоят ли два элемента \textit{в отношении}, или нет. Тогда говорят, что пара $(x, y)$ либо состоит в отношении, либо нет. Записать это можно как $xPy$.
	
	Над бинарными отношениями определены следующие операции:
	\begin{enumerate}
		\item \textit{Пересечение отношений} - обычное пересечение множеств.
		\item \textit{Объединение отношений} - простое объединение множеств.
		\item \textit{Включение} - обычное включение множеств.
		\item \textit{Инверсия} - операция, при которой все пары, которые не были до этого в отношении, в него становятся, и наоборот, все те, которые были в отношении - из него выходят.
		\item \textit{Композиция} - пусть есть два отношения $R \subseteq A \times B$, $S \subseteq B \times C$. Их композицией назовем отношение $R \circ S$, такое что
		\[
		R \circ S = \{(x, y) | \forall z \in B : xRz \land zSy \}
		\]
	\end{enumerate}
	
	\begin{center}
		\line(1,0){450}
	\end{center}

	\section*{24. Свойства бинарных отношений}
	Пусть есть множество $A$ и $P \subseteq A \times A$. Тогда у такого отношения можно рассмотреть возможность наличия свойств:
	\begin{enumerate}
		\item \textit{Рефлексивность}: $\forall a \in A:\ (a, a) \in P$
		\item \textit{Антирефлексивность}: $\forall a \in A:\ (a, a) \notin P$
		\item \textit{Симметричность}: $\forall a, b \in A: (a, b) \in P \Rightarrow (b, a) \in P$
		\item \textit{Антисимметричность}: $\forall a, b \in A: (a, b) \in P \land (b, a) \in P \Rightarrow a = b$
		\item \textit{Транзитивность}: $\forall a, b, c \in A: (a, b) \in P \land (b, c) \in P \Rightarrow (a, c) \in P$
	\end{enumerate}
	
	\begin{center}
		\line(1,0){450}
	\end{center}

	\section*{25. Отношения эквивалентности}
	Бинарное отношение называется \textit{отношением эквивалентности}, если оно рефлексивно, симметрично и транзитивно.
	
	\begin{center}
		\line(1,0){450}
	\end{center}

	\section*{26. Отношения порядка}
	Бинарное отношение может называться \textit{нестрогим частичным порядком}, если оно рефлексивное, антисимметричное и транзитивное.
	
	Бинарное отношение может называться \textit{строгим частичным порядком}, если оно антирефлексивно, антисимметрично и транзитивно.
	
	\textit{Линейный порядок} это частичный порядок с условием связности, то есть
	\[
	\forall a, b \in X \Rightarrow (a, b) \in P \lor (b, a) \in P 
	\]
	
	\section*{27. Соответствия и функции. Образы и прообразы множеств}
	\textit{Соответствием} или \textit{функцией} называется такое отношение двух множеств, при котором элементам одного множества ставится в соответствие элементы другого множества.
	\[
	f \subseteq A \times B
	\]
	\[
	(x, y) \in f \Leftrightarrow f(x) = y
	\]
	
	Существует операция взятия обратного соответствия:
	\[
	f^{-1} = \{(y, x) \ | \ (x, y) \in f \}
	\]
	
	\textit{Образом} элемента $x$ называется такой элемент $y$, что $f(x) = y$. \textit{Образ множества} - множество всех образов элементов множества.
	
	\textit{Прообразом} элемента $y$ называется такой элемент $x$, что $f(x) = y$. \textit{Прообразом множества} называется множество всех прообразов элементов множества.
	
	\begin{center}
		\line(1,0){450}
	\end{center}
	
	\section*{28. Виды функций}
	Соответствие называется \textit{функциональным}, если $\forall (a = b) \Rightarrow f(a) = f(b)$.
	
	Соответствие называется \textit{всюду определенным}, или \textit{тотальным}, если $\forall x \  \exists y: f(x) = y$.
	
	Соответствие называется \textit{сюрьективным}, если $\forall y \ \exists x: f(x) = y$.
	
	Соответствие называется \textit{инъективным}, если $f(x_1) = f(x_2) \Rightarrow x_1 = x_2$
	
	Соответствие называется \textit{биекцией}, если оно одновременно и сюръекция, и инъекция.
	
	\begin{center}
		\line(1,0){450}
	\end{center}

	\section*{29. Композиция функций, ее свойства}
	\textit{Композицией функций} $f \circ g$ называется функция $f(g(x))$.
	
	\begin{center}
		\line(1,0){450}
	\end{center}

	\section*{30. Обратная функция, ее свойства}
	\textit{Обратная функция $f^{-1}(x)$} - это такая функция, что $f(f^{-1}(x)) = x$.
	
	Если $f$ - инъективна, то $f^{-1}$ - функциональна.
	
	Если $f$ - сюрьективна, то $f^{-1}$ - тотальна.
	
	Если $f^{-1}$ - сюрьективна, то $f$ - инъективна. 
	
	\begin{center}
		\line(1,0){450}
	\end{center}
\end{document}
  