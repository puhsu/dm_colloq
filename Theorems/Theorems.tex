\documentclass[a4paper,12pt]{article}

%% Начало шапки

%% Настройка поддержки русского языка
\usepackage{cmap}                   % Поиск по кириллице
\usepackage{mathtext}               % Кириллица в формулах
\usepackage[T1,T2A]{fontenc}        % Кодировки шрифтов
\usepackage[utf8]{inputenc}         % Кодировка текста
\usepackage[english,russian]{babel} % Подключение поддержки языков

%% Настройка размеров полей
\usepackage[top=0.7in, bottom=0.75in, left=0.625in, right=0.625in]{geometry}

%% Математические пакеты
\usepackage{mathtools}              % Тот же amsmath, только с некоторыми поправками
\usepackage{amssymb}                % Математические символы
\usepackage{amsthm}                 % Оформление теорем
\usepackage{amstext}                % Текстовые вставки в формулы
\usepackage{amsfonts}               % Математические шрифты
\usepackage{icomma}                 % "Умная" запятая: $0,2$ --- число, $0, 2$ --- перечисление
\usepackage{enumitem}               % Для выравнивания itemize (\begin{itemize}[align=left])
\usepackage{array}                  % Таблицы и матрицы
\usepackage{multirow}

%% Алгоритмические пакеты и их настройки
\usepackage{algorithm}              % Шапка алгоритма
\usepackage{algorithmicx}           % Написание алгоритмов
\usepackage[noend]{algpseudocode}   % Написание псевдокода; убраны end
\usepackage{listings}               % Для кода на каком-либо языке программиования
\renewcommand{\algorithmicrequire}{\textbf{Ввод:}}              % Ввод
\renewcommand{\algorithmicensure}{\textbf{Вывод:}}              % Вывод
\floatname{algorithm}{Алгоритм}                                 % Название алгоритма
\renewcommand{\algorithmiccomment}[1]{\hspace*{\fill}\{// #1\}} % Комментарии
\newcommand{\algname}[1]{\textsc{#1}}                           % Вызов функции в алгоритме

\newcommand*{\hm}[1]{#1\nobreak\discretionary{}
	{\hbox{$\mathsurround=0pt #1$}}{}}

%% Шрифты
\usepackage{euscript}               % Шрифт Евклид
\usepackage{mathrsfs}               % \mathscr{}

%% Графика
\usepackage[pdftex]{graphicx}       % Вставка картинок
\graphicspath{{images/}}            % Стандартный путь к картинкам
\usepackage{tikz}                   % Рисование всего
\usepackage{pgfplots}               % Графики
\usetikzlibrary{calc,matrix}

%% Прочие пакеты
\usepackage{indentfirst}                    % Красная строка в начале текста
\usepackage{epigraph}                       % Эпиграфы
\usepackage{fancybox,fancyhdr}              % Колонтитулы
\usepackage[colorlinks=true, urlcolor=blue]{hyperref}   % Ссылки
\usepackage{titlesec}                       % Изменение формата заголовков
\usepackage[normalem]{ulem}                 % Для зачёркиваний
\usepackage[makeroom]{cancel}               % И снова зачёркивание (на этот раз косое)

%% Прочее
\mathtoolsset{showonlyrefs=true}        % Показывать номера только у тех формул,
% на которые есть \eqref{} в тексте.
\renewcommand{\headrulewidth}{1.8pt}    % Изменяем размер верхнего отступа колонтитула
\renewcommand{\footrulewidth}{0.0pt}    % Изменяем размер нижнего отступа колонтитула

%Прочее
\usepackage{forest} % Деревья

\renewcommand{\Re}{\mathrm{Re\:}}
\renewcommand{\Im}{\mathrm{Im\:}}
\newcommand{\Arg}{\mathrm{Arg\:}}
\renewcommand{\arg}{\mathrm{arg\:}}
\newcommand{\Mat}{\mathrm{Mat}}
\newcommand{\M}{\mathrm{M}}
\newcommand{\id}{\mathrm{id}}
\newcommand{\isom}{\xrightarrow{\sim}} 
\newcommand{\leftisom}{\xleftarrow{\sim}}
\newcommand{\Hom}{\mathrm{Hom}}
\newcommand{\Ker}{\mathrm{Ker}\:}
\newcommand{\rk}{\mathrm{rk}\:}
\newcommand{\diag}{\mathrm{diag}}
\newcommand{\ort}{\mathrm{ort}}
\newcommand{\pr}{\mathrm{pr}}
\newcommand{\vol}{\mathrm{vol\:}}
\newcommand{\Tr}{\mathrm{tr\:}}
\newcommand{\sgn}{\mathrm{sgn\:}}
\newcommand{\al}{\alpha}

%% Определения
\newtheorem{definition}{Определение}
\newtheorem*{defin}{Определение}
\newtheorem{Def}{Определение}
\newtheorem*{Lemma}{Лемма}
\newtheorem{Suggestion}{Предложение}
\newtheorem*{Examples}{Пример}
\newtheorem*{Consequence}{Следствие}
\newtheorem{Theorem}{Теорема}
\newtheorem*{theorem}{Теорема}
\newtheorem{Statement}{Утверждение}
\newtheorem*{Task}{Упражнение}
\newtheorem*{Designation}{Обозначение}
\newtheorem*{Generalization}{Обобщение}
\newtheorem*{Thedream}{Предел мечтаний}
\newtheorem*{Properties}{Свойства}
\newtheorem*{Note}{Замечание}

\newtheorem*{ther}{Теорема}
\newtheorem*{Induction_def} {Принцип математической индукции}
\newtheorem*{Full_induction_def} {Принцип полной математической индукции}
\newtheorem*{Evc} {Алгорим Евклида}
\newtheorem*{state}{Утверждение}
\newtheorem*{statement}{Формулировка}
\newtheorem*{Statements}{Утверждение}

\newcommand{\Z}{\mathbb{Z}}
\newcommand{\N}{\mathbb{N}}
\newcommand{\Q}{\mathbb{Q}}
\newcommand{\R}{\mathbb{R}}
\renewcommand{\C}{\mathbb{C}}
\renewcommand{\L}{\mathscr{L}}
\renewcommand{\epsilon}{\varepsilon}
\renewcommand{\phi}{\varphi}
\newcommand{\e}{\mathbb{e}}
\renewcommand{\l}{\lambda}
\newcommand{\E}{\mathbb{E}}
\def\eps{\varepsilon}
\def\limref#1#2{{#1}\negmedspace\mid_{#2}}
\newcommand{\vvector}[1]{\begin{pmatrix}{#1}_1 \\\vdots\\{#1}_n\end{pmatrix}}
\renewcommand{\vector}[1]{({#1}_1, \ldots, {#1}_n)}

\definecolor{Gray}{gray}{0.9}
\newcolumntype{g}{>{\columncolor{Gray}}c}

\newtheorem*{canonther}{Теорема о приведении матрицы к каноническому виду}
\newtheorem*{lem}{Лемма}
\newtheorem*{lem1}{Лемма 1}
\newtheorem*{lem2}{Лемма 2}
\newtheorem*{lem3}{Лемма 3}

\newcommand{\p}{^{\prime}}

\begin{document}
	\title{Материалы для подготовки к коллоквиуму\\ по дискретной математике \\
	Теоремы}
	\author{ПМИ 2016 \\ Орлов Никита, Рубачев Иван, Ткачев Андрей, Евсеев Борис}
	\maketitle
	
	\section*{}

	
	\begin{center}
		\line(1,0){450}
	\end{center}
    \newpage
	\section{1. Вывод принципа полной математической индукции из принципа математической индукции}

	\begin{Induction_def}
		Если для утверждения зависящего от положительного натурального $n$ выполняются следущие условия:
		\begin{itemize}
			\item 1. Утверждение истинно при $n = 1$
			\item 2. Когда утверждение истинно при $n = k$, оно истинно и при $n = k + 1$
		\end{itemize}
		Тогда утверждение истинно при всех положительных $n$.
	\end{Induction_def}

	\begin{Full_induction_def}
		Если для утверждения зависящего от положительного натурального $n$ выполняются следущие условия:
		\begin{itemize}
			\item 1. Утверждение истинно для $n = 1$
			\item 2. Если утверждение истинно для всех $n \leqslant k$, оно также истинно и для $n = k + 1$
		\end{itemize}
		Тогда утверждение истинно при всех положительных $n$.
	\end{Full_induction_def}
	\begin{state} 
		Если уместна математическая индукция, то уместна и сильная индукция.
	\end{state}
	\begin{proof}
		В дальньейших рассуждениях будем считать, что $n$ - натуральное, большее или равное $1$, а также обозначим утверждение зависящее от $n$ за $\phi(n)$.

		Предположим, что для $\phi(n)$ выполняются условия $(1)$ и $(2)$ для сильной индукции.

		Пусть $\psi(k) \Leftrightarrow $ <<$\phi(n)$ истинно для всех $n \leqslant k$>>.

		Попытаемся доказать, что утверждение $\psi(n)$ истинно для всех положительных натуральных $n$ по индукции. Как следствие, мы получим, что и $\phi(n)$ верно для всех положительных $n$, т.е. тот же вывод, который должен дать принцип сильной индукции.

		\textit{База}. В силу нашего предположения $\phi(1)$ истинно (гипотеза $(1)$ сильной индукции верна), но тогда истинно и $\psi(1)$, по опеределению $\psi(n)$.

		\textit{Предположение}. Пусть верно $\psi(k)$.

		\textit{Шаг}. Мы предположили, что для $\phi(n)$ выполняются гипотезы сильной индукции, а значит, если <<$\phi(n)$ верно для всех $n \leqslant k$>>, то и $\phi(k + 1)$ - верно. По предположению индукции - $\psi(k) \Rightarrow \phi(k + 1)$ (см. определение $\psi(n)$ и гипотезу $(2)$ сильной индукции). Получаем, что $\psi(k + 1)$ - истинно, т.к. $\phi(n)$ истинно для всех $n \leqslant k + 1$ $\Rightarrow$ $\psi(k + 1)$.

		Согласно принципу мат. индукции $\psi(k)$ - верно для всех положительных $k$, занчит утверждение <<$\phi(n)$ истинно для всех $n \leqslant k$>> верно при всех $k$, а значит $\phi(n)$ - верно для всех $n$.

		Таким образом, из принципа мат. индукции следует принцип полной мат. индукциию.  
 	\end{proof}

 	\section*{2. Бином Ньютона. Формула для биномиальных коэффициентов}
    Число сочетаний из $n$ по $k$ равно:
    \[
        \binom{n}{k} = \frac{n!}{k!(n - k)!}
    \]
    \begin{proof}
        На первое место можно поставить любой из $n$ элементов, на второе любой из $n - 1$ оставшихся, $\ldots$, на $k$-е любой из $n - k + 1$. Тогда по правилу произведения существует $n(n - 1)(n - 2)\cdots(n - k + 1)$ упорядоченных наборов. Но порядок нам не важен, поэтому существует $\displaystyle\frac{n(n - 1)(n - 2)\cdots(n - k + 1)}{k!} = \frac{n!}{k!(n - k)!}$ неупорядоченных наборов.
    \end{proof}
    Формула бинома Ньютона имеет вид:
    \[
        (a + b)^{n} = \binom{n}{0}a^{n} + \binom{n}{1}a^{n - 1}b + \ldots + \binom{n}{k}a^{n - k}b^{k} + \ldots + \binom{n}{n}b^{n} = \sum_{k = 0}^{n}\binom{n}{k}a^{n - k}b^{k}
    \]
    \begin{proof}
        Раскрытие скобок даст все возможные комбинации $a$ и $b$ длины $n$. Так как умножение коммутативно, то элементы с одинаковым количеством $b$ можно сгрупировать. Тогда перед $a^{n - k}b^{k}$ будет стоять коэффициент $c$. Количество слогаемых, в которых $b$ встречается ровно $k$ раз равно $\displaystyle\binom{n}{k}$. Тогда $c = \displaystyle\binom{n}{k}$, а значит:
        \[
            (a + b)^{n} = \sum_{k = 0}^{n}\binom{n}{k}a^{n - k}b^{k}
        \]
    \end{proof}

    \section*{4. Задача Муавра (решение уравнения $x_1+\ldots+x_m=k$) }
    \begin{Statements}
    Число решений уравнения $x_1 + x_2 + \ldots + x_k = n$ в неотрицательных целых
    числах равно ${n+k-1 \choose k-1}$
    \end{Statements}

    \begin{proof}
        \par Воспользуемся методом <<шаров и перегородок>>. Пусть есть $n$ шаров и
        $k-1$ перегородок, тогда какая-то их расстановка однозначно задаёт
        решение уравнения: $x_1$ --  количество шаров перед первой
        перегородкой, $x_2$ -- между 1 и 2, и так далее, количество шаров после
        последней перегородки - $x_k$. Тогда число решений равно ${n+k-1
        \choose k-1}$.
        \par Докажем справедливость данной формулы. Рассмотрим $n$ одинаковых
        объектов, добавим к ним ещё $k-1$ таких же объектов. Тогда, заменив
        какие-то $k-1$ объектов на перегородки, мы получим разбиение множества
        из $n$ элементов на $k$ непересекающихся подмножеств.
    \end{proof}

 	\section*{5. Доказательство формулы включений и исключений}
 	\begin{defin} [Формула включений и исключений.] Формула включений-исключений — комбинаторная формула, позволяющая определить мощность объединения конечного числа конечных множеств, которые в общем случае могут пересекаться друг с другом.
 	\end{defin}
 	\begin{state}
		Пусть $ A_{1}, A_{2},\ldots , A_{n} $ — конечные множества. Формула включений-исключений утверждает:
		$$\biggl | \bigcup_{i=1}^{n}A_i \biggl | =
		 \sum_{i} | A_i | - \sum_{i<j} | A_i \cap A_j | + \sum_{i<j<k} | A_i \cap A_j \cap A_k | - \ldots + (-1)^{n-1} | A_1 \cap A_2 \cap \ldots \cap A_n |.$$
 	\end{state}

 	\begin{proof}
	 	Рассмотрим произвольный элемент $x \in \biggl | \bigcup_{i=1}^{n}A_i \biggl |$, входящий в ровно $S$ множеств $A_{q_1}, ... A_{q_S}$ и подсчитаем, сколько раз он учитывается в правой части формулы включений-исключений (вернее покажем, что учитывается ровно 1 раз):
	 		\begin{itemize}
	 			\item В первой сумме $\sum_{i} | A_i |$ элемент  $x$ посчитан ровно ${S \choose 1} = S$ раз (В слагаемых $A_{q_1}, ... A_{q_S}$).

	 			\item Во второй сумме $\sum_{i<j} | A_i \cap A_j |$ элемент $x$ посчитан ровно ${S \choose 2}$ раз (количесво попарных пересечений $A_i \cap A_j$, таких, что $A_i, A_j \in A_{q_1}, ... A_{q_S}$).

	 			\item В третьей сумме $\sum_{i<j<k} | A_i \cap A_j \cap A_k |$ $x$ будет посчитан ${S \choose 3}$ раза (количество пересечений $A_i \cap A_j \cap A_k$ для которых $i, j \in q_1, \ldots q_S$).

	 			...

	 			\item В $S$-ой сумме $\sum_{i_1<i_2<\ldots<i_S} | A_{i_1} \cap A_{i_2} \cap \ldots \cap A_{i_S} |$
	 			$x$ будет посчитан ${S \choose S} = 1$  раз ($x$ войдет только в слагаемое $| A_1 \cap A_2 \cap \ldots \cap A_n |$).

	 			\item суммы, содержащие $S + 1$ и более пересечений, не учитывают элемент $x$, поскольку $x$ не входит в пересечение более чем $S$ множеств.
	 		\end{itemize}

	 		Таким образом $x$ оказывается посчитанным ровно $S - {S \choose 2} + {S \choose 3} - \ldots + (-1)^{S + 1} {S \choose S}$ раз. Покажем, что эта сумма в точности равна 1. Воспользуемся биномом Ньютона:

	 		$$0 = (1 - 1)^S =
	 		\sum_{k = 0}^{S} {S \choose k} \cdot 1^{S - k} \cdot (-1)^{k} =
	 		1 - \sum_{k = 1}^{S} {S \choose k} \cdot 1^{S - k} \cdot (-1)^{k + 1}$$

	 		$$\Updownarrow$$
	 		$$1 = \sum_{k = 1}^{S} {S \choose k} \cdot (-1)^{k + 1} =
	 		S - {S \choose 2} + {S \choose 3} - \ldots + (-1)^{S + 1} {S \choose S}$$

	 		Таким образом, каждый $x \in \biggl | \bigcup_{i=1}^{n}A_i \biggl |$ учитывается и левой и правой частью формулы ровно 1 раз, и очевидно, что все прочие $y \notin \biggl | \bigcup_{i=1}^{n}A_i \biggl |$ не учитываются ни правой, ни левой частями.
 	\end{proof}


    \section*{6. Формулы для суммы степеней вершин в неориентированном и в ориентированном графе}

    \begin{defin}
        Сумма степеней всех вершин в неориентированном графе равна удвоенному числу ребер. $\displaystyle\sum_{v \in V(G)} \deg(v) = 2 \cdot |E(G)|$
    \end{defin}
    \begin{proof}
        Пусть в графе степень каждой вершины равна $0$ (в графе нет ребер). При добавлении ребра, связывающего любые две вершины, сумма всех степеней увеличивается на 2 единицы. Таким образом, сумма всех степеней вершин четна и равна удвоенному числу ребер.
    \end{proof}
    \begin{defin}
        Число исходящих степеней вершин равно числу входящих, равно числу ребер.
    \end{defin}

        \section*{8. Критерий двураскрашиваемости графа.}
    \begin{Statements}
        Неориентированный граф является 2-раскрашиваемым тогда и только тогда,
        когда в нём нет циклов нечётной длины.
    \end{Statements}

    \begin{proof}
        \par $\Rightarrow$
        Пусть в графе есть цикл нечётной длины. Покрасим какую-то вершину цикла
        в первый цвет и будем двигаться по нему в одном направлении, крася
        каждую следующую вершину в противоположный цвет. Тогда, вернувшись в
        исходную вершину, получим противоречие.
        \par $\Leftarrow$
        Пусть циклов нечётной длины нет. Выберем произвольную вершину $A$ и
        покрасим её в первый цвет. Для любой другой вершины $B$ рассмотрим
        количество рёбер в пути $A\rightarrow B$.
        \par Если есть два пути $A\rightarrow B$ таких, что в одном чётное число
        рёбер, а в другом -- нечётное, то есть цикл с нечётным числом рёбер,
        который получается, если пройти $A\rightarrow B$ по первому пути и
        вернуться $B \rightarrow A$ по второму.
        \par Следовательно, между любыми двумя вершинами все пути либо чётной,
        либо нечётной длины. Раскрасить граф можно следующим образом:
        \begin{itemize}
            \item выделим остовное дерево, раскрасим корень в первый цвет
            \item раскрасим его потомков во второй цвет
            \item для каждого из потомков раскрасим всех его потомков опять в
                первый цвет, и.т.д
        \end{itemize}
        Полученная раскраска будет корректной, так как в остовном дереве
        любой путь между вершинами одного цвета имеет чётную длину (по
        построению), а по доказанному выше путей нечётной длины между такими
        вершинами нет.
    \end{proof}

    \section*{10. Деревья -- это в точности минимально связные графы}
    \begin{proof}\ \\
        $[\Rightarrow]$ Докажем индукцией по числу вершин. База: для $n = 2$ существует лишь одно дерево, для которого утверждение очевидно. Предположим это для некоторого дерева $G_{n}$ на $n$ вершинах, в котором $n - 1$ ребро. Шаг для $n + 1$: добавляя одну вершину $u$, нужно связать её с графом $G_{n}$, то есть соединить с некоторыми вершинами. Если бы мы соединили её с двумя вершинами $v_{1}$ и $v_{2}$, то у нас в графе $G_{n+1}$ получился бы цикл, так как в $G_{n}$ уже существовал путь $v_{1}, a_{1}, a_{2},\ldots, a_{k}, v_{2}$, а значит в $G_{n+1}$ существует цикл $v_{1}, a_{1}, a_{2},\ldots, a_{k}, v_{2}, u, v_{1}$, а значит $G_{n+1}$ -- не дерево. Значит, при добавлении вершины мы можем добавить не более одного ребра (а для сохранения связности ещё и более 0), значит $G_{n+1}$ должен содержать $n - 1 + 1 = n$ рёбер, что означает, что предположение индукции выполнено и для $n + 1$.
        
        \noindent$[\Leftarrow]$ Для начала докажем что в связном графе не может меньше чем $n - 1$ ребро по индукции. База: для $n = 2$ граф на 2-ух вершинах, все очевидно. Шаг для $n + 1$: если для $n$ вершин утверждение верно, то для $n + 1$ вершины оно тоже будет верно, так как нужно связать добавленную вершину как минимум с одним ребром (то есть ребер станет не менее чем $n - 1 + 1 = n$). Пусть у нас есть связный граф на $n$ вершинах, с $n - 1$ ребрами и в этом графе есть циклы. Из некоторого цикла удалим ребро соединявшее вершины $u$ и $v$, при этом граф останется связным, но в нем будет уже $n - 2$ ребра -- получили противоречие. Значит в таком минимально связном графе нет циклов, то есть этот граф -- дерево.
    \end{proof}

    \section*{12. Эквивалентность определений дерева и графа с простым путём
    между любыми двумя вершинами.}
    \begin{Statements}
        Деревья это в точности графы, в которых для любых двух вершин есть ровно
        один простой путь с концами в этих вершинах.
    \end{Statements}

    \begin{proof}
        \par $\Rightarrow$
        \par По определению дерева оно является связным графом без циклов.
        Рассмотрим какие-то две вершины $a$ и $b$. Докажем, что существует
        ровно один простой путь между ними.
        \par Поскольку дерево по определению связно, путь есть. Докажем его
        единственность.
        \par Если есть несколько путей, то маршрут из $a$ в $b$ по первому пути
        и обратно по другому пути будет являться циклом -- значит, путь только
        один.
        %\par Подвесим дерево за вершину $a$ и будем спускаться вниз от неё,
        %каждый раз выбирая то поддерево, которое содержит $b$.

        \par $\Leftarrow$
        \par Рассмотрим две вершины $a$ и $b$ данного графа, по условию между
        ними существует простой путь. Если таких путей несколько, то маршрут из
        $a$ в $b$ по первому пути и обратно по другому пути будет являться
        циклом. Следовательно, путей не более одного. Если же такого пути нет,
        то вершина $b$ не достижима из $a$, то есть граф не связен.
        Следовательно, такой граф является деревом.
    \end{proof}

 	\section*{13. Существование остовного дерева}
 	\begin{defin} \textbf{Частичный граф} исходного графа 
 		$G = (V, E)$ — граф $G^{\prime} = (V, E^{\prime})$, $E^{\prime} \subseteq E$.
	\end{defin}

	\begin{defin} \textbf{Остовное дерево} связного графа $G = (V, E)$ — всякий его частичный граф, являющийся деревом.
	\end{defin}

	\begin{Lemma} 
		Если граф связен, то у него есть остовное дерево. 
	\end{Lemma}

	\begin{proof}
	   Для начала докажем вспомогательную лемму:
		\begin{Lemma} Если граф связен и содержит хотябы один цикл, то из него можно удалить ребро не нарушая связности. 
		\end{Lemma}

		\begin{proof}[Доказательство леммы]
		Пусть $ G = (V, E) $ и цикл в нем: $ u_0 \rightarrow u_1 \rightarrow ... u_n \rightarrow u_0,\ u_i \in V$. Поймем, что если удалить любое ребро принадлежащее цикул, связность не нарушится. Покажем в частности, что можно удалит ребро $(u_0, u_1)$. Действительно, если есть какой-нибудь путь из $v  \in V$ в $w \in V$, проходящий через ребро $ (u_0,\ u_1) $, то существует путь проходящий через прочие ребра цикла, ведь в цикле до каждой вершины можно дойти хотя бы двумя разными путями, значит удаление ребра не изменит того факта, что $v$ соединено путем с $w$. Если пути из $v$ к $w$ не содержат ребра $(u_0, u_1)$, то очевидно, что его удаление на их связи не отразится $\Rightarrow$ граф без этого ребра останется связанным. Тогда удалим его и получим связный граф.
		\end{proof}

		Пусть тепереь $G = (V,\ E)$ - связный граф, для которого нужно доказать существование остовног дерева. Возможны два сценария:
			\begin{enumerate} 
			\item Граф $G$ - связный граф без циклов.
			\item В графе $G$ есть хотя бы один цикл. 
			\end{enumerate}

		В первом случае $G$ - дерево по определению, а значит сам является своим остовным деревом.

		Во втором случае, по доказанной лемме, мы можем удалить из $G$ ребро не нарушая связности. Так сделаем же это. Если полученный граф - цикличен, то снова удалим ребро не нарушая связности, иначе остановимся и порадуемся; индуктивно будем повторять описанные операции, на каждой иттерации имея связный граф; число ребер в графе - конечно, значит процесс не может продолжаться вечно $\Rightarrow$ в какой-то момент мы не сможем удалить ребро не нарушая связности, что было бы не возможно, если бы в графе остался цикл. В ходе описанных операций мы не добавляли новых ребер и не удаляли вершин $\Rightarrow$ если  $G^{\prime} = (V^{\prime},\ E^{\prime})$ - итоговый граф, то $V^{\prime} = V$, $E^{\prime} \subseteq E$ $\Rightarrow$ $G^{\prime}$ - частичный граф графа $G$, связный и без цилов, т.е. дерево $\Rightarrow$ $G^{\prime}$ по определению - остовное дерево графа $G$.
	\end{proof}

    \section*{14. Равносильность совойств ориентированных графов\ldots}
    \begin{statement}
        Следующие свойства ориентированных графов равносильны:
        \begin{enumerate}
            \item Каждая компонента сильной связности состоит из одной вершины.
            \item Вершины графа можно занумеровать так, чтобы каждое ребро вело из вершины с меньшим номером в вершину с большим номером.
            \item В графе нет циклов длины больше $1$.
        \end{enumerate}
        \begin{proof}
            Рассмотрим вершины занумерованные таким образом. Из того, что номера все время возрастают следует отсутствие циклов в графе, так как в вершину с меньшим номером нельзя попасть из вершины с большим номером
        \end{proof}
    \end{statement}

    \section*{16. Критерий Дирака гамильтоновости графа.}
    \begin{Statements}
        Критерий Дирака: граф $G$ на $n$ вершинах содержит гамильтонов цикл,
        если каждая вершина графа имеет степень не меньшую, чем $\frac{n}{2}$.
    \end{Statements}
    \begin{proof}
        Рассмотрим самую длинную простую цепь в графе, обозначим её
        $x_1\rightarrow x_2\rightarrow\ldots\rightarrow x_m$. Докажем, что
        существует вершина $x_i$ такая, что $x_i\rightarrow x_m$ и $x_{i+1}
        \rightarrow x_1$.
        \par Выберем из множества вершин этой цепи два подмножества номеров
        вершин ($1\leq i \leq m-1$):
        \begin{itemize}
            \item множество вершин из цепи, соединённых с последней вершиной
                $x_m$, то есть $A=\{i|(x_i, x_m)\in E\}$
            \item множество вершин из цепи, соединённых со первой вершиной
                $x_1$, то есть $B=\{i|(x_1, x_{i+1}\in E\}$
        \end{itemize}
        Все соседние с вершиной $x_m$, находятся среди $x_1\ldots x_{m-1}$, так
        как в противном случае существует некая вершина $x_k$ вне цепи и данная
        цепь не является самой длинной. Так как  по условию  степень вершнины $x_1\geq\frac{n}{2}$ , 
        то и $|A|\geq\frac{n}{2}$, аналогично
        $|B|\geq\frac{n}{2}$. \par
        Тогда $|A|+|B|\geq n$, но по построению элементы данных множеств – это числа 
        $1\leq i \leq m-1$, это означает, что множества пересекаются и у них есть некоторый 
        общий элемент $j$. Таким образом, в графе имеются ребра $x_1x_{j+1}$ и  $x_mx_{j}$
        Тогда рассмотрим цепь $x_{1}\rightarrow
        x_{2}\rightarrow \ldots\rightarrow x_{j} \rightarrow x_{m} \rightarrow
        x_{m-1}\rightarrow \ldots \rightarrow x_{j+1} \rightarrow x_1$, то есть простой цикл на $m$ вершинах.
        \par Если существует некая вершина вне этой цепи, то данная цепь не
        является самой длинной. Следовательно, в ней присутствуют все вершины
        из графа, то есть $m=n$, а найденый цикл является гамильтоновым.
    \end{proof}

	\section*{17. Сравнение $ax \equiv 1\ (mod\ N)$ имеет решение тогда и только тогда, когда $(a, N) = 1$}
		\begin{Note} Здесь и далее условимся обозначать НОД$(a, N)$, как $(a,\ N)$.
		\end{Note}
		\begin{state}
		Сравнение $ax \equiv 1\ (mod N)$ имеет решение $(1)$ $\Leftrightarrow$ $(a, N) = 1$ $(2)$. 
		\end{state}
		\begin{proof}
		Докажем следствие $(1) \Rightarrow (2)$
		$$ax - 1\equiv 0\ (mod\ N)$$ 
		$$\Downarrow$$ 
		$$N | (ax - 1)$$
		$$\Downarrow$$ 
		$$(ax - 1) = Nk,\ k \in Z.$$

		Пусть $(a,\ N) = b$ ($1 \leqslant b$, т.к. 1 - всегда делитель).
		Тогда $a = a\p \cdot b$, $N = N\p \cdot b$ $\Rightarrow$
		$$a\p bx - 1 = N\p bk$$
		$$\Downarrow$$
		$$1 = b(a\p x - N\p k)$$

		По определению $b|1$, но тогда $|b| \leqslant 1$, но тогда $b = 1$ $\Rightarrow$$(a,\ N) = 1$.

		Докажем следствие $(2) \Rightarrow (1)$: $(2) \Rightarrow (a,\ N) = 1$, тогда по соотношению Безу $\exists m,\ k:\ am + Nk = 1 \Rightarrow am = 1 - Nk \Rightarrow$ $am \equiv 1\ (mod\ N)$, и $x=m$ - решение сравния $ax \equiv 1\ (mod\ N)$.  
		\end{proof}

		\section*{18. Признаки делимости на 3, 9 и 11}

        Число $x$ делистся на $3$ (на $9$) тогда и только тогда, когда сумма его цифр делится на $3$ (на $9$)
        \begin{proof}
            Пусть $x = \overline{a_{n}a_{n - 1}\ldots a_{1}a_{0}} = 10^{n}a_{n} + 10^{n - 1}a_{n - 1} + \ldots + 10a_{1} + a_{0}$. Так как\\
            $10 \equiv 1 \pmod 3$, то:
            \[
                x \equiv \sum_{i = 0}^{n} a_{i} \pmod 3
            \]
            Для делимости на $9$ доказательство аналогично.
        \end{proof}
        Число $x$ делится на $11$, тогда и только тогда, когда:
        \[
        	11 | \bigg(\sum_{2 \mid i}^{n} a_{i} - \sum_{2 \nmid i}^{n} a_{i}\bigg)
        \]
        \begin{proof}
        	$10 \equiv -1 \pmod{11}$, значит $10^{n} \equiv (-1)^{n} \pmod{11}$. Тогда:
        	\[
        		x \equiv (-1)^{n}a_{n} + (-1)^{n - 1}a_{n - 1} + \ldots + (-1)a_{1} + a_{0} \equiv \sum_{2 \mid i}^{n} a_{i} - \sum_{2 \nmid i}^{n} a_{i} \pmod{11}
        	\]
        \end{proof}

        \section*{20. Теорема Эйлера}
        \begin{ther}
            Пусть $N$ -- произвольное простое число, $\phi(N)$ -- функция Эйлера
            (то есть число остатков от $0$ до $N-1$), а число $a$ -- один из этих
            остатков, взаимно простой с $N$. Тогда:
            $$
                a^{\phi(N)}\equiv 1 \bmod N
            $$
        \end{ther}
        \begin{proof}
            Поскольку $a$ взаимно просто с $N$ и $x_{i}$ взаимно просто с $N$, то и
            $x_{i}\cdot a$ также взаимно просто с $N$, то есть существует $x_j$
            такой, что $x_{i}a\equiv x_{j}\bmod N$.\par
            Отметим, что все остатки $x_i\cdot a$ различны по модулю $N$. Пусть это
            не так, тогда $x_{i_1}a\equiv x_{i_2}a \bmod N \Rightarrow
            a(x_{i_1}-x_{i_2}) = 0$, то есть $x_{i_1}\equiv x_{i_2} \bmod N$ -- это
            противоречит тому, что все остатки $x_1\ldots x_{\phi(N)}$ различны.
            \par Перемножим все сравнения $x_i\cdot a \equiv x_j\bmod N$, получим
            $$
                x_1 \cdots x_{\phi(N)} a^{\phi(N)} \equiv x_1 \cdots
                x_{\varphi(N)}\bmod N\\
                x_1 \cdots x_{\phi(N)} (a^{\phi(N)}-1) \equiv 0 \bmod N
            $$
            Поскольку каждый из остатков $x_1 \ldots x_{\phi(N)} $ взаимно
            прост с $N$, можно записать:

            $$
                a^{\phi(N)}-1 \equiv 0 \bmod N
            $$
        \end{proof}

		\section*{21. Корректность алгоритма Евклида и расширенного алгоритма Евклида.}
		\begin{Evc}
		Пусть $a$ и $b$ - целые числа одноверемненно не равные нулю, и последовательность чисел
		$ x_0 > x_1 > x_2 > x_3 \cdots > x_n > 0$
		определена тем, что $x_0 = a,\ x_1 = b$, каждое $x_k,\ k > 1$ — это остаток от деления предпредыдущего числа на предыдущее, а предпоследнее делится на последнее нацело, то есть:
		$$a = x_0q_1 + x_1,$$
		$$b = x_1q_2 + x_2,$$
		$$x_2 = x_3q_3 + x_4,$$
		$$\cdots$$
		$$x_{k-2} = x_{k-1} q_{k-1} + x_k,$$	 
		$$\cdots$$
		$$x_{n-2} = x_{n-1}q_{n-1}+ x_n,$$
		$$x_{n-1} = x_n q_n.$$
		Тогда $(a, b)$ равен $x_n$, последнему ненулевому члену этой последовательности.
		\end{Evc}
		\begin{proof}
			Поймем, что такие $x_1, x_2, x_3, x_4, \cdots x_n$ - существуют, причем единственно: всегда можно найти остаток $m$ (причем единственным образом) при делении $x_k$ на $x_{k + 1}$, если $x_{k + 1} \ne 0$, причем $a > b > r_k > x_{k + 1} > m$, т.е. каждый следующий член последовательности строго меньше предыдущего, но т.к. числа ее составляющие - целые, то убывать бесконечно она не может, а значит $\exists x_{n + 1} = 0$ - последний член последовательности.

			Докажем тогда, что если $x_n$ - последний не нулевой член последовательности, то $(a, b) = (x_n, 0) = x_n \ne 0$. Для этого заметим две вещи:
			\begin{enumerate}
				\item $r \ne 0 \Rightarrow (r, 0) = |r|$ так как 0 делится на любое целое число, кроме нуля.

				\item Пусть $a = bq + r$, тогда $(a,\ b) = (b,\ r)$.
				Пусть $k$ — любой общий делитель чисел $a$ и $b$, не обязательно наибольший, тогда $a = t_1k$ и $b = t_2k$, где $t_1$ и $t_2$ — целые числа из определения.

				Тогда $k$ является также общим делителем чисел $b$ и $r$, так как $b$ делится на $k$ по определению, а $r = a - b\cdot q = (t_1 - t_2\cdot q)\cdot k$ (выражение в скобках есть целое число, следовательно, $k$ делит $r$ без остатка). 
				
				Обратное также верно. Любой делитель $k$ чисел $b$ и $r$ так же является делителем $a$ и $b$: $a = b \cdot q + r = k\cdot (b\p q + r\p)$$\Rightarrow$ $k|a$.
				
				Следовательно, все общие делители пар чисел $a$, $b$ и $b$, $r$совпадают. Другими словами, нет общего делителя у чисел $a$, $b$, который не был бы также делителем $b$, $r$, и наоборот.
				
				В частности, наибольший общий делитель остается тем же самым. Что и требовалось доказать.
			\end{enumerate}

			Тогда по построению последовательности $\{x_i\}: (x_0,\ x_1) = (x_1,\ x_2) = (x_2,\ x_3) = \ldots = (x_n,\ 0) = x_n$.
		\end{proof}
		\begin{Evc}[Расширенный алгоритм Евклида] 
			Формулы для $x_i$ могут быть переписаны следующим образом:
			$$x_0 = aq_0 + bp_0,$$
			$$x_1 = aq_1 + bp_1,$$
			$$x_2 = aq_2 + bp_2,$$
			$$x_3 = aq_3 + bp_3,$$
	 		$$\vdots$$
			$$(a,\ b) = x_n = as + bt$$

            Т.е. НОД$(a,\ b)$ можно представить в виде $ax + by$, где $x,\ y$ - какие-то целые числа.
		\end{Evc}
		\begin{proof}
			Докажем по индукции по $n$. 

			\textit{База.} $x_0 = a + b \cdot 0$, $x_1 = a \cdot 0 + b$. Т.е. $q_0 = P_1 = 1,\ p_0 = q_1 = 0$

			\textit{Предположение.} Пусть $x_{k - 2} = aq_{k - 2} + bp_{k - 2}$ и $x_{k - 1} = aq_{k - 1} + bp_{k - 1}$.

			\textit{Шаг.} Докажем, что $x_k = aq_{k} + bp_{k}$, где $q_k,\ p_k$ - целые. Мы помним, что $x_k$ - остаток от деления $x_{k - 2}$ на $x_{k - 1}$, значит по определнию: $m\cdot x_{k - 1} + x_k = x_{k - 2}$, где $m$ - какое-то целое число. Тогда $x_k = x_{k - 2} - m \cdot x_{k - 1}$, по п.и., $x_k = aq_{k - 2} + bp_{k - 2} - m(aq_{k - 1} + bp_{k - 1}) = a(q_{k - 2} - mq_{k - 1}) + b(p_{k - 2} - mp_{k - 1}) = aq_{k} + bp_{k}$.

			Таким образом каждое из чисел $x_i$ представимо в виде линейной комбинации $a$ и $b$ (В частности, если $(a,\ b) = 1$, то $\exists x,\ y: ax + by = 1$).
		\end{proof}

        \section*{22. Основная теорема арифметики}
        \begin{Lemma}
            Если простое число $p$ делит без остатка произведение двух целых чисел $x \cdot y$, то $p$ делит $x$ или $y$.
        \end{Lemma}
        \begin{proof}
            Пусть $x \cdot y$ делятся на $p$, но $x$ не делится на $p$, тогда $x$ и $p$ -- взаимнопростые, следовательно, найдутся такие целые числа $u$ и $v$, что:
            \[
                x\cdot u + p \cdot v = 1
            \]
            Умножая обе части на $y$ получаем:
            \[
                (x\cdot y) \cdot u  + p \cdot v \cdot y = y
            \]
            Здесь оба слогаемых в левой части делятся на $p$, значит и $y$ делится на $p$. 
        \end{proof}
        \begin{theorem}
            Каждое натуральное число $n >1 $ представляется в виде $n=p_1\cdot\ldots\cdot p_k$, где $p_1,\ldots,p_k$ -- простые числа, причём такое представление единственно с точностью до порядка следования сомножителей.
        \end{theorem}
        \begin{proof}\ \\
        \textbf{Существование.} Пусть $n$ -- наименьшее целое число не разложимое в произведение простых чисел. Оно не может быть единицей по формулировке теоремы. Оно не может быть и простым, потому что любое простое число является произведением одного простого числа -- себя. Если $n$ составное, то оно -- произведение двух меньших натуральных чисел. Каждое из них можно разложить в произведение простых чисел, значит $n$ тоже является произведением простых чисел. Противоречие.

        \noindent\textbf{Единственность.} Пусть $n$ -- наименьше натуральное число, разложимое в произведение простых чисел двумя разными способами. Если оба разложения пустые — они одинаковы. В противном случае, пусть $p$ -- любой из сомножителей в любом из двух разложений. Если $p$ входит и в другое разложение, мы можем сократить оба разложения на $p$ и получить два разных разложения числа $\displaystyle\frac{n}{p}$, что невозможно. А если $p$ не входит в другое разложение, то одно из произведений делится на $p$, а другое -- не делится (как следствие из леммы), что противоречит условию.
        \end{proof}

        \section*{23. Китайская теорема об остатках}
        \begin{theorem}
            Для любых попарно взаимно-простых $a_1, a_2, ..., a_n$ и для любых $r_1, r_2, ..., r_n$ таких, что $0 \leqslant r_i < a_i$, существует и единственен с точностью до операции взятия по модулю $M = \prod_{1}^{n} a_i$ $x$ являющийся решением системы (1): 
            \begin{equation*} 
                   \begin{cases}
                   x \equiv r_1 \mod a_1 \\
                   x \equiv r_2 \mod a_2 \\
                   \vdots
                   \\
                   x \equiv r_n \mod a_n
                   \end{cases}
            \end{equation*}
            
            И любой $x^\prime \equiv x \mod \prod_{1}^{n} a_i$ так же является решением этой системы.

            (Иная формулировка: Если натуральные числа $ a_{1},a_{2},\dots ,a_{n}$  
            попарно взаимно просты, то для любых целых $ r_{1},r_{2},\dots ,r_{n}$ таких, что 
            $ 0\leqslant r_{i}<a_{i}$  при всех $ i\in \{1,2,\dots ,n\},$  найдётся число $N$, 
            которое при делении на $ a_{i}$  даёт остаток $ r_{i}$  при всех $ i\in \{1,2,\dots ,n\}$ . Более того, если найдутся два таких числа $N_{1}$  и $N_{2}$ , то 
            $N_1 \equiv N_2 \pmod{a_1\cdot a_2\cdot \ldots\cdot a_n}$.)
        \end{theorem}
        \begin{proof}
            Покажем, что $ x=\sum _{i=1}^{n}r_{i}M_{i}M_{i}^{-1}$ (2), 
            где $ M_{i}={\frac {M}{a_{i}}}$ , а $ M_{i}^{-1} $  - обратный к $ M_{i}$  элемент по модулю $a_{i}$, является решением указанной выше системы.

            Проверим, что для него выполняется i-е равенство в системе: 
            $$ x\equiv \sum _{j=1}^{n} r_{j} M_{j} M_{j}^{-1}
                \equiv r_{i}M_{i} M_{i}^{-1}
                \equiv r_{i}{\pmod {a_{i}}} $$
            Второе равенство справедливо т.к.
            $ M_{j}\equiv \prod_{k\neq j}^{n} a_{k}
                    \equiv 0{\pmod {a_{i}}}$ 
            при всех $ i\neq j$ (т.е. все слагаемые кроме $j$-ого делятся на $a_j$), третье т.к. $ M_{i}^{-1} $ является обратным для $ M_{i} $ по модулю $a_{i} $. Повторяя рассуждения для всех $i$, убедимся, что $x$, определенный формулой (2), является решением для (1).

            В силу выбранного числа $M$ все числа $x\p\equiv x{\pmod  M}$ будут удовлетворять системе.

            Покажем теперь, что среди чисел $0,\ 1, \dots,\ M-1$ (множество $A$) не найдется другого решения кроме найденного нами ранее. 
            Проведем доказательство этого факта от противного. 
            Предположим, что получилось найти хотя бы два решения $x_{1},\ x_{2}\in A$ для некоторого набора остатков $r$. Так как множество $B$ всех допустимых наборов 
            $ (r_{1},\ r_{2},\dots ,r_{n}) $ является равномощным множеству $ A $ (количество наборов остатков в $B$: $|B| = a_1 \cdot a_2 \ldots = M = |A|$ ), 
            то для $\overline A_{x}:=A\setminus \{x_{1},x_{2}\} $ 
            и 
            $ \overline B_{r}:=B\setminus \{r\}$ выполнено 
            $|\overline A_{x}|<|\overline B_{r}|$. Однако по доказанному ранее, 
            для любого набора из $ \overline B_{r} $ существует решение из $ \overline A_{x}$, следовательно по принципу Дирихле найдутся как минимум 2 набора остатков, которым соответствует одно и то же $x\in A$. Для такого $ x $ найдется $ a_{i} $ такое, что 
            $x\equiv r_{1},\ x\equiv r_{2}{\pmod  {a_{i}}}$ и $r_{1}\neq r_{2}$. Противоречие.

        \end{proof}

        \section*{24. Мультипликативность функции Эйлера. Формула для функции Эйлера}
        \begin{Statements}
            Для взаимно простых $m$ и $n$ верно, что $\phi(mn) = \phi(m)\phi(n)$
        \end{Statements}
        \begin{proof}

        \end{proof}

        \section*{25. Доказательство корректности определения классов эквивалентности}
        \begin{theorem}
            Для любого отношения эквивалентности на множестве $A$ множество классов эквивалентности образует разбиение множества $A$. Обратно, любое разбиение множества $A$ задает на нем отношение эквивалентности, для которого классы эквивалентности совпадают с элементами разбиения.
        \end{theorem}

        \begin{proof}
            \textit{Докажем прямое следствие.}

            Каждому $x \in A$ сопоставим $[x] = \{y|\ x\thicksim y\}$ - пожмножетсво множество всех элементов с которыми $x$ вступает в отношение $\thicksim $.

            Утверждается, что система подмножеств $[x]$ образует разбиение $A$. Действительно, во-первых, каждое подмножество $[x] \ne \emptyset$ , так как в силу рефлексивности отношения $\thicksim $  $x \in [x]$.   

            Во-вторых,  два различных подмножества $[x]$ и $[y]$ не имеют общих элементов. Рассуждая от противного, допустим существование элемента $z$ такого, что $z \in [x]$ и $z \in [y]$. Тогда $z\thicksim x$ и $z\thicksim y$. Поэтому для любого элемента $t \in [x]$ из $t\thicksim x$, $z\thicksim x$ и $z\thicksim y$ в силу симметричности и транзитивности отношения a вытекает $aPy$ ($t\thicksim x$ и $x\thicksim z$ $\Rightarrow$ $tPz$, но $z\thicksim y$ $\Rightarrow$ $t\thicksim y$), то есть $a \in [y]$. Следовательно, $[x] \subseteq [y]$. Аналогично получаем, что $[y] \subseteq [x]$. Полученные два включения влекут равенство $[x] = [y]$, противоречащее предположению о несовпадении подмножеств $[x]$ и $[y]$. Таким образом, $[x] \cap [y] = \emptyset$.

            В-третьих, объединение всех подмножеств $[x]$ (классов эквивалентности) совпадает со множеством $A$, ибо для любого элемента $x \in A$ выполняется условие $x \in [x]$.

            Итак, система подмножеств эквивалентности $[x]$, образует разбиение множества $A$.

            \textit{Обратное следтвие.}

            Пусть есть разбиение $A$ на непересекающиеся множества $M_0, \ldots, M_1$. Тогда отношение эквивалентности на $A$ задается так:

            $$a \thicksim b \leftrightarrow (a \in M_i \wedge b \in M_i)$$

             Свойства транзитивности, рефлексивности и симметричности очевидны ( Например для транзитивности: $a \thicksim b$ и $b \thicksim c$, значит $(a \in M_i \wedge b \in M_i) \wedge (b \in M_i \wedge c \in M_i) \Leftrightarrow (a \in M_i \wedge c \in M_i) \Leftrightarrow a \thicksim c$). Тогда, два элемента принадлежат одному классу тогда и только тогда, когда они лежат в одном подмножестве $M_i$, т.е. классы задаются разбиением.
        \end{proof}
        \section*{26. Критерий того, что бинарное отношение записывается с помощью функции полезности}
        \begin{statement}
            Пусть множество $A$ конечно, тогда соотношение:
            \[
                xPy \iff u(x) > u(y)
            \]
            Выполняется для некоторой функции $u(x)$ в том и только в том случае, когда $P$ -- отношение слабого порядка.
        \end{statement}
        \begin{proof}\ 
            \\$[\Rightarrow]$ Докажем это утверждение в одну сторону. Пусть выполняется данное соотношение. Для того чтобы доказать, что $P$ -- отношение слабого порядка, необходимо проверить его антирефлексивность, транзитивность и транзитивность его дополнения.

            \textbf{Антирефлексивность.} Пусть $x \in A$. Тогда $u(x)$ не больше $u(x)$, то есть $x\overline{P}x$. Значит отношение $P$ антирефлексивно.

            \textbf{Транзитивность.} Пусть $x,y,z \in A$, таковы, что $xPy$ и $yPz$. Это значит, что $u(x) > u(y)$ и $u(y) > u(z)$. Следовательно, $u(x) > u(z)$, или $xPz$, значит $P$ транзитивно.

            \textbf{Транзитивность дополнения.} Пусть $x,y,z \in A$ таковы, что $x\overline{P}y$ и $y\overline{P}z$. В силу соотношения из формулировки $u(x) \leqslant u(y)$ и $u(y) \leqslant u(z)$, отсюда $x\overline{P}z$, то есть $\overline{P}$ транзитивно.
            \newline\ 

            \noindent$[\Leftarrow]$ Пусть $P$ -- слабый порядок. Определим значение $u(x)$, как число элементов во множестве $\{y|xPy\}$, то есть число альтернатив, которые менее предпочтительны, чем $x$. Докажем, что при этом $xPy \iff u(x) > u(y)$.

            Пусть $xPy$. Поскольку отношение $P$ транзитивно, то для любого $z$, такого, что $yPz$, верно и $xPz$. Поэтому из $x$ выходят дуги как минимум в те же вершины, что и из $y$, значит $u(x) \geqslant u(y)$. Кроме того $P$ антирефлексивно, поэтому из $y$ не ведет дуга в $y$, а из $x$ в $y$ ведет. Значит, $u(x) > u(y)$.

            Обратно, пусть $u(x) > u(y)$, т.е. из $x$ выходит больше дуг, чем из $y$. Значит, существует такой элемент $z$, что $xPz$, но $y\overline{P}z$. Если $x\overline{P}y$, то отношение $\overline{P}$ не транзитивно, что противоречит условию, значит $(x,y) \in P$.
        \end{proof}

        \section*{28. Теорема о представлении частичного порядка в виде пересечения линейных}
        \begin{Theorem}
            Любой частиный порядок, определенный на множестве из $n$ элементов, можно предстваить, как пересечение не более, чем $n^2$ линейных порядков.
        \end{Theorem}
        
        \begin{proof}
        
            Пусть у нас есть частичный порядок $P$. Рассмотрим несравнимую пару $x$~и~$y$. Образуем новый частичный порядок $P'$, полученный из $P$ добавлением сравимости $xPy$
            и некоторых других для того, чтобы транзитивность сохранилась. Образуем еще один частичный порядок $P''$, полученный из $P$ добавлением сравнимости $yPx$ и некоторых
            других сравнимостей для сохранения тразитивности. Тогда каждый из этих двух частичных порядков можем достроить до линейного порядка (по теореме Шпильрайна).
            Назовем их $Lin_{p'}$ и $Lin_{p''}$ соответсвенно.
            
            Теперь оценим количество несравнимых пар. Всего пар в отношении может быть $n \cdot n = n^2$ штук, однако нас не интересует порядок элементов внутри пар, 
            тогда без учета порядка их  \textit{не более} $2 \cdot \frac{n^2}{2!} = \frac{n^2}{2}$. Получаем, что несравнимых пар также не более $\frac{n^2}{2}$. Тогда рассмотрим для каждой из них
            $Lin_{p'}$ и $Lin_{p''}$, таких линейных порядков в сумме не более $\frac{n^2}{2} = n^2$. Теперь изучим, что будет, если их пересечь. В действительности, мы получим как раз $P$,
            так как если $xPy$, то она принадлежит и $Lin_{p'}$, и $Lin_{p''}$, иначе она будет принадлежать только одному из них, и тогда при пересечении её уже не будет.
            
            $P.S.$ Внимательный читатель скажет, что мы рассмотрели только случай, когда нам нужно получить строгий частичный порядок, однако на самом деле получение нестрогого
            обходится нам <<дёшево>> и не влияет на нашу оценку, так как её можно осуществить параллельно с другими пересечениями.
            
           
        \end{proof}

        \section*{29. Критерий существования функции, обратной к данной. Критерий биекции в терминах обратной функции}

        \subparagraph{Критерий существования функции, обратной к данной.}
        Пусть $f$ - функциональное соответствие $f:X \rightarrow Y$. Тогда обратное соответствие: $f^{-1} = {(y,\ x)|\ (x,\ y) \in f \Leftrightarrow f(x) = y}$.

        \begin{Note} $f^{-1}$ - функционально $\Leftrightarrow$ $f$ - инъетивно.
        \end{Note}

        \begin{proof}

        Докажем $\rightarrow$. 

        Из того, что $f^{-1}$ - функционально $\Rightarrow \forall\ y \in Y:\ f^{-1}(y) = x$ и $f^{-1}(y) = x\p \Leftrightarrow x = x\p$ $\Rightarrow$ если $f(x) = f(x\p) = y$, то $f^{-1}(y) = x = x\p$, что и означает инъективность $f$. 

        Докажем $\leftarrow$.

        Из инъективности $f$ $\Rightarrow$ $\forall\ y \in Y:$  $f(x) = y$ и $f(x\p) = y \Leftrightarrow x = x\p \Rightarrow$ если $(y,\ x) \in f^{-1}$ и $(y,\ x\p) \in f^{-1}$, то $x = x\p$ $\Rightarrow$ $f^{-1}$ - функционально.
        \end{proof} 

        \subparagraph{Критерий биекции в терминах обратной функции}
        \begin{theorem} Критерией биективности:
            \begin{enumerate}
                \item Если $f$ - биекция $A \leftrightarrow B$, 
                то $f \circ f^{-1} = id_B$ и
                            $f^{-1} \circ f = id_A$.       
                \item Если $f$ - функция $A \rightarrow B$ и существует $g: B \rightarrow A$, такая что $f \circ g = id_B$ и $g \circ f = id_A$, то $f^{-1} = g$ и $f$ - биекция.    
            \end{enumerate} 
        \end{theorem}

        \begin{proof}
            Утверждение $1$ проверяется непосредственно, по свойствам биекции: $\forall a \in A:\ f^{-1} \circ f(a) = a$ и
                     $\forall b \in B:\ f \circ f^{-1}(b) = b$.

            Докажем 2, проверив $f$ на свойства биекции.

            \textit{Всюду определенность}
            Если $f$ не всюду определена, то  $g \circ f(x) = g(f(x))$ - не всюду определена, а значит не тождествена, что противоречит гипотизе. Значит $f$ - тотальна.

            \textit{Инъективность.} 
            Пусть $f(x_1) = f(x_2) \Rightarrow g(f(x_1)) = g(f(x_2)) \Rightarrow x_1 = x_2$.

            \textit{Сюръективность.} 
            Пусть $f$ не принимает значение $b \in B$, тогда $f(g(\ldots))$ не принимает значение $b$, значит $f \circ g$ - не тождественна, что противоречит условию. Тогда $\forall b \in B:\ \exists a \in A:\ f(a) = b$.

            Таким образом $f$ - биекция. Тогда очевидно, что $g = f^{-1}$ (проверяется поэлементно из композиции $g \circ f = id_A$, $f \circ g = id_B$: $(a,\ b) \in f \Rightarrow (b,\ a) \in g$ и аналогично, если $(b,\ a) \in g$, то $(a,\ b) \in f$).
        \end{proof}

    \section*{30. Биекция между двоичными словами, подмножествами конечного множества и характеристическими функциями}

        \begin{defin}
            Характеристической функцией множества $X \subset U$ называют функцию $\displaystyle\chi_{X}$, которая равна $1$ на элементах $X$ и $0$ на остальных элементах $U$.
        \end{defin}
        Составим двоичное слово следующим образом: если $i$ элемент лежит в $X$, то на $i$-м месте ставим $1$, иначе $0$. Биекция между характеристической функцией и подмножеством очевидна -- значения характеристической функции однозначно задают подмножество.

\end{document}
