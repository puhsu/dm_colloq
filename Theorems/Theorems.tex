\documentclass[a4paper,12pt]{article}

%% Начало шапки

%% Настройка поддержки русского языка
\usepackage{cmap}                   % Поиск по кириллице
\usepackage{mathtext}               % Кириллица в формулах
\usepackage[T1,T2A]{fontenc}        % Кодировки шрифтов
\usepackage[utf8]{inputenc}         % Кодировка текста
\usepackage[english,russian]{babel} % Подключение поддержки языков

%% Настройка размеров полей
\usepackage[top=0.7in, bottom=0.75in, left=0.625in, right=0.625in]{geometry}

%% Математические пакеты
\usepackage{mathtools}              % Тот же amsmath, только с некоторыми поправками
\usepackage{amssymb}                % Математические символы
\usepackage{amsthm}                 % Оформление теорем
\usepackage{amstext}                % Текстовые вставки в формулы
\usepackage{amsfonts}               % Математические шрифты
\usepackage{icomma}                 % "Умная" запятая: $0,2$ --- число, $0, 2$ --- перечисление
\usepackage{enumitem}               % Для выравнивания itemize (\begin{itemize}[align=left])
\usepackage{array}                  % Таблицы и матрицы
\usepackage{multirow}

%% Алгоритмические пакеты и их настройки
\usepackage{algorithm}              % Шапка алгоритма
\usepackage{algorithmicx}           % Написание алгоритмов
\usepackage[noend]{algpseudocode}   % Написание псевдокода; убраны end
\usepackage{listings}               % Для кода на каком-либо языке программиования
\renewcommand{\algorithmicrequire}{\textbf{Ввод:}}              % Ввод
\renewcommand{\algorithmicensure}{\textbf{Вывод:}}              % Вывод
\floatname{algorithm}{Алгоритм}                                 % Название алгоритма
\renewcommand{\algorithmiccomment}[1]{\hspace*{\fill}\{// #1\}} % Комментарии
\newcommand{\algname}[1]{\textsc{#1}}                           % Вызов функции в алгоритме

\newcommand*{\hm}[1]{#1\nobreak\discretionary{}
	{\hbox{$\mathsurround=0pt #1$}}{}}

%% Шрифты
\usepackage{euscript}               % Шрифт Евклид
\usepackage{mathrsfs}               % \mathscr{}

%% Графика
\usepackage[pdftex]{graphicx}       % Вставка картинок
\graphicspath{{images/}}            % Стандартный путь к картинкам
\usepackage{tikz}                   % Рисование всего
\usepackage{pgfplots}               % Графики
\usetikzlibrary{calc,matrix}

%% Прочие пакеты
\usepackage{indentfirst}                    % Красная строка в начале текста
\usepackage{epigraph}                       % Эпиграфы
\usepackage{fancybox,fancyhdr}              % Колонтитулы
\usepackage[colorlinks=true, urlcolor=blue]{hyperref}   % Ссылки
\usepackage{titlesec}                       % Изменение формата заголовков
\usepackage[normalem]{ulem}                 % Для зачёркиваний
\usepackage[makeroom]{cancel}               % И снова зачёркивание (на этот раз косое)

%% Прочее
\mathtoolsset{showonlyrefs=true}        % Показывать номера только у тех формул,
% на которые есть \eqref{} в тексте.
\renewcommand{\headrulewidth}{1.8pt}    % Изменяем размер верхнего отступа колонтитула
\renewcommand{\footrulewidth}{0.0pt}    % Изменяем размер нижнего отступа колонтитула

%Прочее
\usepackage{forest} % Деревья

\renewcommand{\Re}{\mathrm{Re\:}}
\renewcommand{\Im}{\mathrm{Im\:}}
\newcommand{\Arg}{\mathrm{Arg\:}}
\renewcommand{\arg}{\mathrm{arg\:}}
\newcommand{\Mat}{\mathrm{Mat}}
\newcommand{\M}{\mathrm{M}}
\newcommand{\id}{\mathrm{id}}
\newcommand{\isom}{\xrightarrow{\sim}}
\newcommand{\leftisom}{\xleftarrow{\sim}}
\newcommand{\Hom}{\mathrm{Hom}}
\newcommand{\Ker}{\mathrm{Ker}\:}
\newcommand{\rk}{\mathrm{rk}\:}
\newcommand{\diag}{\mathrm{diag}}
\newcommand{\ort}{\mathrm{ort}}
\newcommand{\pr}{\mathrm{pr}}
\newcommand{\vol}{\mathrm{vol\:}}
\newcommand{\Tr}{\mathrm{tr\:}}
\newcommand{\sgn}{\mathrm{sgn\:}}
\newcommand{\al}{\alpha}

%% Определения
\newtheorem{definition}{Определение}
\newtheorem*{defin}{Определение}
\newtheorem{Def}{Определение}
\newtheorem*{Lemma}{Лемма}
\newtheorem{Suggestion}{Предложение}
\newtheorem*{Examples}{Пример}
\newtheorem*{Consequence}{Следствие}
\newtheorem{Theorem}{Теорема}
\newtheorem*{ther}{Теорема}
\newtheorem{Statement}{Утверждение}
\newtheorem*{Statements}{Утверждение}
\newtheorem*{Task}{Упражнение}
\newtheorem*{Designation}{Обозначение}
\newtheorem*{Generalization}{Обобщение}
\newtheorem*{Thedream}{Предел мечтаний}
\newtheorem*{Properties}{Свойства}
\newtheorem*{Note}{Замечание}

\newcommand{\Z}{\mathbb{Z}}
\newcommand{\N}{\mathbb{N}}
\newcommand{\Q}{\mathbb{Q}}
\newcommand{\R}{\mathbb{R}}
\renewcommand{\C}{\mathbb{C}}
\renewcommand{\L}{\mathscr{L}}
\renewcommand{\epsilon}{\varepsilon}
\renewcommand{\phi}{\varphi}
\newcommand{\e}{\mathbb{e}}
\renewcommand{\l}{\lambda}
\newcommand{\E}{\mathbb{E}}
\def\eps{\varepsilon}
\def\limref#1#2{{#1}\negmedspace\mid_{#2}}
\newcommand{\vvector}[1]{\begin{pmatrix}{#1}_1 \\\vdots\\{#1}_n\end{pmatrix}}
\renewcommand{\vector}[1]{({#1}_1, \ldots, {#1}_n)}

\definecolor{Gray}{gray}{0.9}
\newcolumntype{g}{>{\columncolor{Gray}}c}

\newtheorem*{canonther}{Теорема о приведении матрицы к каноническому виду}
\newtheorem*{lem}{Лемма}
\newtheorem*{lem1}{Лемма 1}
\newtheorem*{lem2}{Лемма 2}
\newtheorem*{lem3}{Лемма 3}



\begin{document}
	\title{Материалы для подготовки к коллоквиуму\\ по дискретной математике \\ Теоремы}
	\author{ПМИ 2016 \\ Орлов Никита, Рубачев Иван, Ткачев Андрей, Евсеев Борис}
	\maketitle

	\begin{center}
		\line(1,0){450}
	\end{center}
    \newpage

    \section*{4. Задача Муавра (решение уравнения $xi_1+\ldots+x_m=k$) }
    \begin{Statements}
    Число решений уравнения $x_1 + x_2 + \ldots + x_k = n$ в неотрицательных целых
    числах равно ${n+k-1 \choose k-1}$
    \end{Statements}

    \proof{
        \par Воспользуемся методом <<шаров и перегородок>>. Пусть есть $n$ шаров и
        $k-1$ перегородок, тогда какая-то их расстановка однозначно задаёт
        решение уравнения: $x_1$ --  количество шаров перед первой
        перегородкой, $x_2$ -- между 1 и 2, и так далее, количество шаров после
        последней перегородки - $x_k$. Тогда число решений равно ${n+k-1
        \choose k-1}$.
        \par Докажем справедливость данной формулы. Рассмотрим $n$ одинаковых
        объектов, добавим к ним ещё $k-1$ таких же объектов. Тогда, заменив
        какие-то $k-1$ объектов на перегородки, мы получим разбиение множества
        из $n$ элементов на $k$ непересекающихся подмножеств.
    }

    \section*{8. Критерий двураскрашиваемости графа.}
    \begin{Statements}
        Неориентированный граф является 2-раскрашиваемым тогда и только тогда,
        когда в нём нет циклов нечётной длины.
    \end{Statements}

    \proof{
        \par $\Rightarrow$
        Пусть в графе есть цикл нечётной длины. Покрасим какую-то вершину цикла
        в первый цвет и будем двигаться по нему в одном направлении, крася
        каждую следующую вершину в противоположный цвет. Тогда, вернувшись в
        исходную вершину, получим противоречие.
        \par $\Leftarrow$
        Пусть циклов нечётной длины нет. Выберем произвольную вершину $A$ и
        покрасим её в первый цвет. Для любой другой вершины $B$ рассмотрим
        количество рёбер в пути $A\rightarrow B$.
        \par Если есть два пути $A\rightarrow B$ таких, что в одном чётное число
        рёбер, а в другом -- нечётное, то есть цикл с нечётным числом рёбер,
        который получается, если пройти $A\rightarrow B$ по первому пути и
        вернуться $B \rightarrow A$ по второму.
        \par Следовательно, между любыми двумя вершинами все пути либо чётной,
        либо нечётной длины. Раскрасить граф можно следующим образом:
        \begin{itemize}
            \item выделим остовное дерево, раскрасим корень в первый цвет
            \item раскрасим его потомков во второй цвет
            \item для каждого из потомков раскрасим всех его потомков опять в
                первый цвет, и.т.д
        \end{itemize}
        Полученная раскраска будет корректной, так как в остовном дереве
        любой путь между вершинами одного цвета имеет чётную длину (по
        построению), а по доказанному выше путей нечётной длины между такими
        вершинами нет.
    }

    \section*{12. Эквивалентность определений дерева и графа с простым путём
    между любыми двумя вершинами.}
    \begin{Statements}
        Деревья это в точности графы, в которых для любых двух вершин есть ровно
        один простой путь с концами в этих вершинах.
    \end{Statements}

    \proof{
    \par $\Rightarrow$
        \par По определению дерева оно является связным графом без циклов.
        Рассмотрим какие-то две вершины $a$ и $b$. Докажем, что существует
        ровно один простой путь между ними.
        \par Поскольку дерево по определению связно, путь есть. Докажем его
        единственность.
        \par Если есть несколько путей, то маршрут из $a$ в $b$ по первому пути
        и обратно по другому пути будет являться циклом -- значит, путь только
        один.
        %\par Подвесим дерево за вершину $a$ и будем спускаться вниз от неё,
        %каждый раз выбирая то поддерево, которое содержит $b$.

    \par $\Leftarrow$
        \par Рассмотрим две вершины $a$ и $b$ данного графа, по условию между
        ними существует простой путь. Если таких путей несколько, то маршрут из
        $a$ в $b$ по первому пути и обратно по другому пути будет являться
        циклом. Следовательно, путей не более одного. Если же такого пути нет,
        то вершина $b$ не достижима из $a$, то есть граф не связен.
        Следовательно, такой граф является деревом.
    }

    \section*{16. Критерий Дирака гамильтоновости графа.}
    \begin{Statements}
        Критерий Дирака: граф $G$ на $n$ вершинах содержит гамильтонов цикл,
        если каждая вершина графа имеет степень не меньшую, чем $\frac{n}{2}$.
    \end{Statements}
    \begin{proof}
        Рассмотрим самую длинную простую цепь в графе, обозначим её
        $x_1\rightarrow x_2\rightarrow\ldots\rightarrow x_m$. Докажем, что
        существует вершина $x_i$ такая, что $x_i\rightarrow x_m$ и $x_{i+1}
        \rightarrow x_1$.
        \par Выберем из множества вершин этой цепи два подмножества номеров
        вершин ($1\leq i \leq m-1$):
        \begin{itemize}
            \item множество вершин из цепи, соединённых с последней вершиной
                $x_m$, то есть $A=\{i|(x_i, x_m)\in E\}$
            \item множество вершин из цепи, соединённых со первой вершиной
                $x_1$, то есть $B=\{i|(x_1, x_{i+1}\in E\}$
        \end{itemize}
        Все соседние с вершиной $x_m$, находятся среди $x_1\ldots x_{m-1}$, так
        как в противном случае существует некая вершина $x_k$ вне цепи и данная
        цепь не является самой длинной. Тогда в $A$ лежат не меньше,
        чем половина вершин в графе, то есть $|A|\geq\frac{n}{2}$, аналогично
        $|B|\geq\frac{n}{2}$. \par
        Поскольку всего в графе $n$ вершин, пересечение множеств $A$ и $B$
        непусто, то есть найдётся  вершина $x_j$ с номером $j$ такая, что она
        соединена с $x_1$ и $x_m$. Тогда рассмотрим цепь $x_{j+1}\rightarrow
        x_{j+2}\rightarrow \ldots\rightarrow x_{m} \rightarrow x_j \rightarrow
        x_{j-1}\rightarrow \ldots \rightarrow x_2 \rightarrow x_1 \rightarrow
        x_{j+1}$, то есть простой цикл на $m$ вершинах.
        \par Если существует некая вершина вне этой цепи, то данная цепь не
        является самой длинной. Следовательно, в ней присутствуют все вершины
        из графа, то есть $m=n$, а найденый цикл является гамильтоновым.
    \end{proof}


    \section*{20}



    \section*{24}


    \section*{28}

\end{document}
