\documentclass[a4paper,12pt]{article}

%% Начало шапки

%% Настройка поддержки русского языка
\usepackage{cmap}                   % Поиск по кириллице
\usepackage{mathtext}               % Кириллица в формулах
\usepackage[T1,T2A]{fontenc}        % Кодировки шрифтов
\usepackage[utf8]{inputenc}         % Кодировка текста
\usepackage[english,russian]{babel} % Подключение поддержки языков

%% Настройка размеров полей
\usepackage[top=0.7in, bottom=0.75in, left=0.625in, right=0.625in]{geometry}

%% Математические пакеты
\usepackage{mathtools}              % Тот же amsmath, только с некоторыми поправками
\usepackage{amssymb}                % Математические символы
\usepackage{amsthm}                 % Оформление теорем
\usepackage{amstext}                % Текстовые вставки в формулы
\usepackage{amsfonts}               % Математические шрифты
\usepackage{icomma}                 % "Умная" запятая: $0,2$ --- число, $0, 2$ --- перечисление
\usepackage{enumitem}               % Для выравнивания itemize (\begin{itemize}[align=left])
\usepackage{array}                  % Таблицы и матрицы
\usepackage{multirow}

%% Алгоритмические пакеты и их настройки
\usepackage{algorithm}              % Шапка алгоритма
\usepackage{algorithmicx}           % Написание алгоритмов
\usepackage[noend]{algpseudocode}   % Написание псевдокода; убраны end
\usepackage{listings}               % Для кода на каком-либо языке программиования
\renewcommand{\algorithmicrequire}{\textbf{Ввод:}}              % Ввод
\renewcommand{\algorithmicensure}{\textbf{Вывод:}}              % Вывод
\floatname{algorithm}{Алгоритм}                                 % Название алгоритма
\renewcommand{\algorithmiccomment}[1]{\hspace*{\fill}\{// #1\}} % Комментарии
\newcommand{\algname}[1]{\textsc{#1}}                           % Вызов функции в алгоритме

\newcommand*{\hm}[1]{#1\nobreak\discretionary{}
	{\hbox{$\mathsurround=0pt #1$}}{}}

%% Шрифты
\usepackage{euscript}               % Шрифт Евклид
\usepackage{mathrsfs}               % \mathscr{}

%% Графика
\usepackage[pdftex]{graphicx}       % Вставка картинок
\graphicspath{{images/}}            % Стандартный путь к картинкам
\usepackage{tikz}                   % Рисование всего
\usepackage{pgfplots}               % Графики
\usetikzlibrary{calc,matrix}

%% Прочие пакеты
\usepackage{indentfirst}                    % Красная строка в начале текста
\usepackage{epigraph}                       % Эпиграфы
\usepackage{fancybox,fancyhdr}              % Колонтитулы
\usepackage[colorlinks=true, urlcolor=blue]{hyperref}   % Ссылки
\usepackage{titlesec}                       % Изменение формата заголовков
\usepackage[normalem]{ulem}                 % Для зачёркиваний
\usepackage[makeroom]{cancel}               % И снова зачёркивание (на этот раз косое)

%% Прочее
\mathtoolsset{showonlyrefs=true}        % Показывать номера только у тех формул,
% на которые есть \eqref{} в тексте.
\renewcommand{\headrulewidth}{1.8pt}    % Изменяем размер верхнего отступа колонтитула
\renewcommand{\footrulewidth}{0.0pt}    % Изменяем размер нижнего отступа колонтитула

%Прочее
\usepackage{forest} % Деревья

\renewcommand{\Re}{\mathrm{Re\:}}
\renewcommand{\Im}{\mathrm{Im\:}}
\newcommand{\Arg}{\mathrm{Arg\:}}
\renewcommand{\arg}{\mathrm{arg\:}}
\newcommand{\Mat}{\mathrm{Mat}}
\newcommand{\M}{\mathrm{M}}
\newcommand{\id}{\mathrm{id}}
\newcommand{\isom}{\xrightarrow{\sim}} 
\newcommand{\leftisom}{\xleftarrow{\sim}}
\newcommand{\Hom}{\mathrm{Hom}}
\newcommand{\Ker}{\mathrm{Ker}\:}
\newcommand{\rk}{\mathrm{rk}\:}
\newcommand{\diag}{\mathrm{diag}}
\newcommand{\ort}{\mathrm{ort}}
\newcommand{\pr}{\mathrm{pr}}
\newcommand{\vol}{\mathrm{vol\:}}
\newcommand{\Tr}{\mathrm{tr\:}}
\newcommand{\sgn}{\mathrm{sgn\:}}
\newcommand{\al}{\alpha}

%% Определения
\newtheorem{definition}{Определение}
\newtheorem*{defin}{Определение}
\newtheorem{Def}{Определение}
\newtheorem*{Lemma}{Лемма}
\newtheorem{Suggestion}{Предложение}
\newtheorem*{Examples}{Пример}
\newtheorem*{Consequence}{Следствие}
\newtheorem{Theorem}{Теорема}
\newtheorem*{ther}{Теорема}
\newtheorem{Statement}{Утверждение}
\newtheorem*{Task}{Упражнение}
\newtheorem*{Designation}{Обозначение}
\newtheorem*{Generalization}{Обобщение}
\newtheorem*{Thedream}{Предел мечтаний}
\newtheorem*{Properties}{Свойства}
\newtheorem*{Note}{Замечание}

\newcommand{\Z}{\mathbb{Z}}
\newcommand{\N}{\mathbb{N}}
\newcommand{\Q}{\mathbb{Q}}
\newcommand{\R}{\mathbb{R}}
\renewcommand{\C}{\mathbb{C}}
\renewcommand{\L}{\mathscr{L}}
\renewcommand{\epsilon}{\varepsilon}
\renewcommand{\phi}{\varphi}
\newcommand{\e}{\mathbb{e}}
\renewcommand{\l}{\lambda}
\newcommand{\E}{\mathbb{E}}
\def\eps{\varepsilon}
\def\limref#1#2{{#1}\negmedspace\mid_{#2}}
\newcommand{\vvector}[1]{\begin{pmatrix}{#1}_1 \\\vdots\\{#1}_n\end{pmatrix}}
\renewcommand{\vector}[1]{({#1}_1, \ldots, {#1}_n)}

\definecolor{Gray}{gray}{0.9}
\newcolumntype{g}{>{\columncolor{Gray}}c}

\newtheorem*{canonther}{Теорема о приведении матрицы к каноническому виду}
\newtheorem*{lem}{Лемма}
\newtheorem*{lem1}{Лемма 1}
\newtheorem*{lem2}{Лемма 2}
\newtheorem*{lem3}{Лемма 3}



\begin{document}
	\title{Материалы для подготовки к коллоквиуму\\ по дискретной математике \\ Задачи}
	\author{ПМИ 2016 \\ Орлов Никита, Рубачев Иван, Ткачев Андрей, Евсеев Борис}
	\maketitle
<<<<<<< HEAD:cheat_sheet.tex
	 
	\section*{2. Бином Ньютона. Формула для биномиальных коэффициентов}
        Число сочетаний из $n$ по $k$ равно:
        \[
            \binom{n}{k} = \frac{n!}{k!(n - k)!}
        \]
        \begin{proof}
            На первое место можно поставить любой из $n$ элементов, на второе любой из $n - 1$ оставшихся, $\ldots$, на $k$-е любой из $n - k + 1$. Тогда по правилу произведения существует $n(n - 1)(n - 2)\cdots(n - k + 1)$ упорядоченных наборов. Но порядок нам не важен, поэтому существует $\displaystyle\frac{n(n - 1)(n - 2)\cdots(n - k + 1)}{k!} = \frac{n!}{k!(n - k)!}$ неупорядоченных наборов.
        \end{proof}
        Формула бинома Ньютона имеет вид:
        \[
            (a + b)^{n} = \binom{n}{0}a^{n} + \binom{n}{1}a^{n - 1}b + \ldots + \binom{n}{k}a^{n - k}b^{k} + \ldots + \binom{n}{n}b^{n} = \sum_{k = 0}^{n}\binom{n}{k}a^{n - k}b^{k}
        \]
        \begin{proof}
            Раскрытие скобок даст все возможные комбинации $a$ и $b$ длины $n$. Так как умножение коммутативно, то элементы с одинаковым количеством $b$ можно сгрупировать. Тогда перед $a^{n - k}b^{k}$ будет стоять коэффициент $c$. Количество слогаемых, в которых $b$ встречается ровно $k$ раз равно $\displaystyle\binom{n}{k}$. Тогда $c = \displaystyle\binom{n}{k}$, а значит:
            \[
                (a + b)^{n} = \sum_{k = 0}^{n}\binom{n}{k}a^{n - k}b^{k}
            \]
        \end{proof}
    \pagebreak % временно
    \section*{6. Формулы для суммы степеней вершин в неориентированном и в ориентированном графе}

        \begin{defin}
            Сумма степеней всех вершин в неориентированном графе равна удвоенному числу ребер. $\displaystyle\sum_{v \in V(G)} \deg(v) = 2 \cdot |E(G)|$
        \end{defin}
        \begin{proof}
            Пусть в графе степень каждой вершины равна $0$ (в графе нет ребер). При добавлении ребра, связывающего любые две вершины, сумма всех степеней увеличивается на 2 единицы. Таким образом, сумма всех степеней вершин четна и равна удвоенному числу ребер.
        \end{proof}
        \begin{defin}
            Число исходящих степеней вершин равно числу входящих, равно числу ребер.
        \end{defin}
        \begin{proof}
            Первая часть утверждения очевидна. Каждое ребро выходит из одной вершины и входит в другую, поэтому каждое ребро дает одинаковый вклад в суммы исходящих и входящих степеней вершин. Для доказательства второй части утверждения докажем что число ребер равно числу исходящих степеней вершин. Исходящая степень вершины равна числу ребер, которые из нее выходят. Ребро не может выходить более чем из одной вершины, поэтому сумма исходящих степеней вершин равна числу ребер. По транзитивности отношения <<=>> число ребер равно также и сумме исходящих вершин.
        \end{proof}
    \section*{10. Деревья -- это в точности минимально связные графы}
        \begin{proof}\ \\
            $[\Rightarrow]$ Докажем индукцией по числу вершин. База: для $n = 2$ существует лишь одно дерево, для которого утверждение очевидно. Предположим это для некоторого дерева $G_{n}$ на $n$ вершинах, в котором $n - 1$ ребро. Шаг для $n + 1$: добавляя одну вершину $u$, нужно связать её с графом $G_{n}$, то есть соединить с некоторыми вершинами. Если бы мы соединили её с двумя вершинами $v_{1}$ и $v_{2}$, то у нас в графе $G_{n+1}$ получился бы цикл, так как в $G_{n}$ уже существовал путь $v_{1}, a_{1}, a_{2},\ldots, a_{k}, v_{2}$, а значит в $G_{n+1}$ существует цикл $v_{1}, a_{1}, a_{2},\ldots, a_{k}, v_{2}, u, v_{1}$, а значит $G_{n+1}$ -- не дерево. Значит, при добавлении вершины мы можем добавить не более одного ребра (а для сохранения связности ещё и более 0), значит $G_{n+1}$ должен содержать $n - 1 + 1 = n$ рёбер, что означает, что предположение индукции выполнено и для $n + 1$.
            
            \noindent$[\Leftarrow]$ Для начала докажем что в связном графе не может меньше чем $n - 1$ ребро по индукции. База: для $n = 2$ граф на 2-ух вершинах, все очевидно. Шаг для $n + 1$: если для $n$ вершин утверждение верно, то для $n + 1$ вершины оно тоже будет верно, так как нужно связать добавленную вершину как минимум с одним ребром (то есть ребер станет не менее чем $n - 1 + 1 = n$). Пусть у нас есть связный граф на $n$ вершинах, с $n - 1$ ребрами и в этом графе есть циклы. Из некоторого цикла удалим ребро соединявшее вершины $u$ и $v$, при этом граф останется связным, но в нем будет уже $n - 2$ ребра -- получили противоречие. Значит в таком минимально связном графе нет циклов, то есть этот граф -- дерево.
        \end{proof}
    \pagebreak % временно
    \section*{18. Признаки делимости на 3, 9 и 11}

        Число $x$ делистся на $3$ (на $9$) если сумма его цифр делится на $3$ (на $9$)
        \begin{proof}
            Пусть $x = \overline{a_{n}a_{n - 1}\ldots a_{1}a_{0}} = 10^{n}a_{n} + 10^{n - 1}a_{n - 1} + \ldots + 10a_{1} + a_{0}$. Так как\\
            $10 \equiv 1 \pmod 3$, то:
            \[
                x \equiv \sum_{i = 0}^{n} a_{i} \pmod 3
            \]
            Для делимости на $9$ доказательство аналогично.
        \end{proof}
        Число $x$ делится на $11$, если знакопеременная сумма его слогаемых делится на $11$.
=======
	
	\section*{}

	
	\begin{center}
		\line(1,0){450}
	\end{center}
	
>>>>>>> master:Problems/Problems.tex
\end{document}
